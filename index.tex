% Options for packages loaded elsewhere
\PassOptionsToPackage{unicode}{hyperref}
\PassOptionsToPackage{hyphens}{url}
\PassOptionsToPackage{dvipsnames,svgnames,x11names}{xcolor}
%
\documentclass[
  letterpaper,
  DIV=11,
  numbers=noendperiod]{scrreprt}

\usepackage{amsmath,amssymb}
\usepackage{iftex}
\ifPDFTeX
  \usepackage[T1]{fontenc}
  \usepackage[utf8]{inputenc}
  \usepackage{textcomp} % provide euro and other symbols
\else % if luatex or xetex
  \usepackage{unicode-math}
  \defaultfontfeatures{Scale=MatchLowercase}
  \defaultfontfeatures[\rmfamily]{Ligatures=TeX,Scale=1}
\fi
\usepackage{lmodern}
\ifPDFTeX\else  
    % xetex/luatex font selection
\fi
% Use upquote if available, for straight quotes in verbatim environments
\IfFileExists{upquote.sty}{\usepackage{upquote}}{}
\IfFileExists{microtype.sty}{% use microtype if available
  \usepackage[]{microtype}
  \UseMicrotypeSet[protrusion]{basicmath} % disable protrusion for tt fonts
}{}
\makeatletter
\@ifundefined{KOMAClassName}{% if non-KOMA class
  \IfFileExists{parskip.sty}{%
    \usepackage{parskip}
  }{% else
    \setlength{\parindent}{0pt}
    \setlength{\parskip}{6pt plus 2pt minus 1pt}}
}{% if KOMA class
  \KOMAoptions{parskip=half}}
\makeatother
\usepackage{xcolor}
\setlength{\emergencystretch}{3em} % prevent overfull lines
\setcounter{secnumdepth}{5}
% Make \paragraph and \subparagraph free-standing
\ifx\paragraph\undefined\else
  \let\oldparagraph\paragraph
  \renewcommand{\paragraph}[1]{\oldparagraph{#1}\mbox{}}
\fi
\ifx\subparagraph\undefined\else
  \let\oldsubparagraph\subparagraph
  \renewcommand{\subparagraph}[1]{\oldsubparagraph{#1}\mbox{}}
\fi


\providecommand{\tightlist}{%
  \setlength{\itemsep}{0pt}\setlength{\parskip}{0pt}}\usepackage{longtable,booktabs,array}
\usepackage{calc} % for calculating minipage widths
% Correct order of tables after \paragraph or \subparagraph
\usepackage{etoolbox}
\makeatletter
\patchcmd\longtable{\par}{\if@noskipsec\mbox{}\fi\par}{}{}
\makeatother
% Allow footnotes in longtable head/foot
\IfFileExists{footnotehyper.sty}{\usepackage{footnotehyper}}{\usepackage{footnote}}
\makesavenoteenv{longtable}
\usepackage{graphicx}
\makeatletter
\def\maxwidth{\ifdim\Gin@nat@width>\linewidth\linewidth\else\Gin@nat@width\fi}
\def\maxheight{\ifdim\Gin@nat@height>\textheight\textheight\else\Gin@nat@height\fi}
\makeatother
% Scale images if necessary, so that they will not overflow the page
% margins by default, and it is still possible to overwrite the defaults
% using explicit options in \includegraphics[width, height, ...]{}
\setkeys{Gin}{width=\maxwidth,height=\maxheight,keepaspectratio}
% Set default figure placement to htbp
\makeatletter
\def\fps@figure{htbp}
\makeatother
\newlength{\cslhangindent}
\setlength{\cslhangindent}{1.5em}
\newlength{\csllabelwidth}
\setlength{\csllabelwidth}{3em}
\newlength{\cslentryspacingunit} % times entry-spacing
\setlength{\cslentryspacingunit}{\parskip}
\newenvironment{CSLReferences}[2] % #1 hanging-ident, #2 entry spacing
 {% don't indent paragraphs
  \setlength{\parindent}{0pt}
  % turn on hanging indent if param 1 is 1
  \ifodd #1
  \let\oldpar\par
  \def\par{\hangindent=\cslhangindent\oldpar}
  \fi
  % set entry spacing
  \setlength{\parskip}{#2\cslentryspacingunit}
 }%
 {}
\usepackage{calc}
\newcommand{\CSLBlock}[1]{#1\hfill\break}
\newcommand{\CSLLeftMargin}[1]{\parbox[t]{\csllabelwidth}{#1}}
\newcommand{\CSLRightInline}[1]{\parbox[t]{\linewidth - \csllabelwidth}{#1}\break}
\newcommand{\CSLIndent}[1]{\hspace{\cslhangindent}#1}

\KOMAoption{captions}{tableheading}
\makeatletter
\makeatother
\makeatletter
\@ifpackageloaded{bookmark}{}{\usepackage{bookmark}}
\makeatother
\makeatletter
\@ifpackageloaded{caption}{}{\usepackage{caption}}
\AtBeginDocument{%
\ifdefined\contentsname
  \renewcommand*\contentsname{Table of contents}
\else
  \newcommand\contentsname{Table of contents}
\fi
\ifdefined\listfigurename
  \renewcommand*\listfigurename{List of Figures}
\else
  \newcommand\listfigurename{List of Figures}
\fi
\ifdefined\listtablename
  \renewcommand*\listtablename{List of Tables}
\else
  \newcommand\listtablename{List of Tables}
\fi
\ifdefined\figurename
  \renewcommand*\figurename{Figure}
\else
  \newcommand\figurename{Figure}
\fi
\ifdefined\tablename
  \renewcommand*\tablename{Table}
\else
  \newcommand\tablename{Table}
\fi
}
\@ifpackageloaded{float}{}{\usepackage{float}}
\floatstyle{ruled}
\@ifundefined{c@chapter}{\newfloat{codelisting}{h}{lop}}{\newfloat{codelisting}{h}{lop}[chapter]}
\floatname{codelisting}{Listing}
\newcommand*\listoflistings{\listof{codelisting}{List of Listings}}
\makeatother
\makeatletter
\@ifpackageloaded{caption}{}{\usepackage{caption}}
\@ifpackageloaded{subcaption}{}{\usepackage{subcaption}}
\makeatother
\makeatletter
\@ifpackageloaded{tcolorbox}{}{\usepackage[skins,breakable]{tcolorbox}}
\makeatother
\makeatletter
\@ifundefined{shadecolor}{\definecolor{shadecolor}{rgb}{.97, .97, .97}}
\makeatother
\makeatletter
\makeatother
\makeatletter
\makeatother
\ifLuaTeX
  \usepackage{selnolig}  % disable illegal ligatures
\fi
\IfFileExists{bookmark.sty}{\usepackage{bookmark}}{\usepackage{hyperref}}
\IfFileExists{xurl.sty}{\usepackage{xurl}}{} % add URL line breaks if available
\urlstyle{same} % disable monospaced font for URLs
\hypersetup{
  pdftitle={Emprendimiento con Datos en Turismo},
  pdfauthor={Francisco J. Navarro-Meneses, Editor},
  colorlinks=true,
  linkcolor={blue},
  filecolor={Maroon},
  citecolor={Blue},
  urlcolor={Blue},
  pdfcreator={LaTeX via pandoc}}

\title{Emprendimiento con Datos en Turismo}
\author{Francisco J. Navarro-Meneses, Editor}
\date{2023-06-12}

\begin{document}
\maketitle
\ifdefined\Shaded\renewenvironment{Shaded}{\begin{tcolorbox}[interior hidden, frame hidden, boxrule=0pt, borderline west={3pt}{0pt}{shadecolor}, breakable, enhanced, sharp corners]}{\end{tcolorbox}}\fi

\renewcommand*\contentsname{Table of contents}
{
\hypersetup{linkcolor=}
\setcounter{tocdepth}{2}
\tableofcontents
}
\bookmarksetup{startatroot}

\hypertarget{prefacio}{%
\chapter*{Prefacio}\label{prefacio}}
\addcontentsline{toc}{chapter}{Prefacio}

\markboth{Prefacio}{Prefacio}

Lorem ipsum dolor sit amet, consectetur adipiscing elit. Praesent
dapibus ut libero nec semper. In quam lorem, rutrum in nulla quis,
elementum volutpat odio. Phasellus felis nunc, semper eu sodales eu,
laoreet nec arcu. Sed ac magna quis sapien accumsan gravida. Etiam
tristique dui id elit egestas condimentum. Integer gravida fermentum
placerat. Ut hendrerit viverra ipsum, id vehicula magna tincidunt sed.
Nunc fermentum diam purus, non dignissim purus tincidunt vitae. Sed
tincidunt tortor vitae malesuada molestie. Sed aliquet, est a vulputate
aliquet, massa erat hendrerit arcu, in ultrices nulla nisl vel tortor.
Nam velit est, venenatis et tortor ac, rhoncus feugiat est. Nunc non
neque erat. Etiam eget ipsum fermentum risus lobortis consequat sit amet
sit amet dui. Maecenas auctor vehicula volutpat.

\hypertarget{titular}{%
\section*{Titular}\label{titular}}
\addcontentsline{toc}{section}{Titular}

\markright{Titular}

Etiam ultricies magna imperdiet nunc malesuada, ac lobortis sapien
rhoncus. Aenean eros lectus, accumsan vitae faucibus a, aliquam ac diam.
Maecenas finibus justo non nibh pharetra aliquam. Maecenas egestas, ex
vitae blandit convallis, leo mauris ornare nunc, porta fermentum tellus
est et urna. Aenean eu est et sapien laoreet placerat. Nunc turpis
ipsum, dapibus a ipsum at, tempus pretium lacus. In libero turpis,
tristique id orci a, auctor rhoncus risus. Praesent lacus nunc,
sollicitudin quis ipsum id, pulvinar venenatis mi.

\hypertarget{titular-1}{%
\subsection*{Titular}\label{titular-1}}
\addcontentsline{toc}{subsection}{Titular}

Vivamus libero leo, accumsan non turpis vel, eleifend sodales mauris.
Donec pellentesque, tellus a lacinia vestibulum, nisl lacus dapibus
nibh, id tincidunt turpis erat non nisl. Donec ornare imperdiet metus,
eu sollicitudin risus elementum non. Aliquam metus eros, maximus
dignissim pellentesque nec, venenatis vitae purus. Vivamus laoreet mi
magna, ac malesuada tellus condimentum in. Integer nulla lectus, finibus
sit amet ex nec, finibus molestie diam. Pellentesque gravida ac erat sit
amet tempus.

This is a Quarto book.

\bookmarksetup{startatroot}

\hypertarget{introducciuxf3n}{%
\chapter*{Introducción}\label{introducciuxf3n}}
\addcontentsline{toc}{chapter}{Introducción}

\markboth{Introducción}{Introducción}

\begin{itemize}
\tightlist
\item
  The development of technology and the mounting use of information
  communication technology (ICT) have an important role for the business
  organisations and operations
\item
  Although the need of having a reliable ICT is apparent, there are many
  business organisations that do not adopt this technology. For a large
  business organisation with ample human and financial resources, the
  adoption of ICT may not be a significant problem. However, for
  small-sized business organisations which face resources limitations,
  the ICT adoption becomes a problem.
\item
  This issue turns out to be more severe for small businesses in
  developing countries such as Indonesia, in which its entrepreneurs
  face education and cultural constraints (Anggadwita et al., 2015;
  Tambunan, 2011)
\item
  This is not a book about studying the adoption of ICT and data by
  entrepreneurs, but about how to help them
\end{itemize}

\hypertarget{benefits-and-types-of-data}{%
\section*{Benefits and Types of Data}\label{benefits-and-types-of-data}}
\addcontentsline{toc}{section}{Benefits and Types of Data}

\markright{Benefits and Types of Data}

\begin{itemize}
\tightlist
\item
  ICT could help the business organisation to achieve a greater
  efficiency and lower cost in their operation.
\item
  Impact of data on business productivity, cost, revenue, and
  profitability (Gërguri-Rashiti et al., 2017).
\item
  Data and ICT to push innovation and business performance (Ramadani et
  al., 2016).
\item
  From a consumer perspective, ICT, especially the internet, has an
  impact on the way consumers purchase products and services.
\item
  The Internet more reliable and affordable for many customers. It is
  now easier for the consumer to find information and vendors of a
  product or service, not only from the local market, but also from the
  global market.
\item
  Having well developed and operationalised ICT is an important factor
  to build the competitive business advantage (Gërguri-Rashiti et al.,
  2017).
\end{itemize}

\hypertarget{index}{%
\subsection*{Index}\label{index}}
\addcontentsline{toc}{subsection}{Index}

\begin{enumerate}
\def\labelenumi{\arabic{enumi}.}
\tightlist
\item
  Context and Background
\end{enumerate}

\begin{itemize}
\tightlist
\item
  Entrepreneurship and tourism
\item
  Entrepreneurs, data and ICT
\item
  Characteristics of the entrepreneurial firm
\end{itemize}

\begin{enumerate}
\def\labelenumi{\arabic{enumi}.}
\setcounter{enumi}{1}
\tightlist
\item
  Understanding data
\end{enumerate}

\begin{itemize}
\tightlist
\item
  What is data?
\item
  Benefits of data
\item
  Types of Data
\item
  The Data life-cycle
\item
  Management of data in small enterprises
\end{itemize}

\begin{enumerate}
\def\labelenumi{\arabic{enumi}.}
\setcounter{enumi}{2}
\tightlist
\item
  The Entrepreneurial data-driven process
\end{enumerate}

\begin{itemize}
\tightlist
\item
  The classic vs data-driven entrepreneurial process
\end{itemize}

\begin{enumerate}
\def\labelenumi{\arabic{enumi}.}
\setcounter{enumi}{3}
\tightlist
\item
  Methods and Technologies
\end{enumerate}

\begin{itemize}
\tightlist
\item
  Deployment methods
\item
  Cloud and entrepreneurs
\item
  Low code
\item
  Big data for entrepreneurs
\end{itemize}

\begin{enumerate}
\def\labelenumi{\arabic{enumi}.}
\setcounter{enumi}{4}
\tightlist
\item
  Data considerations for Tourism Entrepreneurs
\end{enumerate}

\begin{itemize}
\tightlist
\item
  Ética de Datos y Emprendimiento
\item
  Finanzas de Datos para Emprendedores
\item
  Conocimiento del Cliente a través de los Datos
\item
  La Gobernanza de Datos en PYMEs Emprendedoras
\end{itemize}

\begin{enumerate}
\def\labelenumi{\arabic{enumi}.}
\setcounter{enumi}{5}
\tightlist
\item
  Starting-up a data-driven entrepreneur (Wrapping all up)
\end{enumerate}

\begin{itemize}
\tightlist
\item
  Stages
\item
  Leadership styles
\end{itemize}

\begin{enumerate}
\def\labelenumi{\arabic{enumi}.}
\setcounter{enumi}{6}
\tightlist
\item
  What's next
\end{enumerate}

\begin{itemize}
\tightlist
\item
  Role of AI
\item
  IoT
\item
  Big Data and analytics
\end{itemize}

\bookmarksetup{startatroot}

\hypertarget{turismo-datos-y-emprendimiento}{%
\chapter{Turismo, Datos y
Emprendimiento}\label{turismo-datos-y-emprendimiento}}

Autor: Cai Liang Zhou Cheng

\hypertarget{turismo-y-emprendimiento}{%
\section{Turismo y Emprendimiento}\label{turismo-y-emprendimiento}}

El turismo, como bien es sabido, es una actividad transversal y de
alcance global en el que el desplazamiento de personas a distintos
destinos turísticos alrededor del globo con diversos motivos de viaje
divergentes a las del tan clásico viaje por ocio o vacaciones es objeto
de caracterización de acuerdo con una definición más actual, fidedigna y
sesgada. Esto quiere decir que el propósito del desplazamiento puede ser
tan válido el experimentar nuevas culturas lejos de tu lugar de
residencia como el de tener un viaje de negocios por asuntos
profesionales. Por tanto, es una industria masificada en constante
expansión que amplias fronteras y que implica propiamente dicho a un
gran número de participantes, sectores y sistemas los cuales deberán
estar en constante interdependencia para que como tal, el turismo fluya
adecuadamente dando beneficio mutuo a todas sus partes.

Sin embargo, en los últimos años, ha surgido una interesante relación
entre el turismo y el emprendimiento como consecuencia de la gran
creación de oportunidades que permite generar un fenómeno tan
ventralmente heterogéneo, pero a la vez efectivo y cargado de potencial
desarrollo e innovación como lo es el turismo. El emprendimiento en el
ámbito turístico en este sentido, aunque no estrictamente, se refiere a
la creación de nuevas empresas, productos, servicios, procesos,
soluciones, innovaciones, investigaciones, desarrollos o experiencias
que buscan satisfacer, cubrir, gestionar y tratar las necesidades y
demandas de los turistas modernos, exponencialmente más exigentes y
difíciles de fidelizar, con un perfil definido alrededor de la era
tecnología y con un carácter volátil al ser complejo de predecir. Los
emprendedores en esta industria identifican oportunidades, innovan en la
forma en que se ofrecen los servicios y utilizan estrategias creativas
para diferenciarse de la competencia y sobre todo de aquellos modelos de
negocio tradicionales llevados por empresas consolidadas con una rica
historia remarcada detrás.

Esta relación entre el turismo y el emprendimiento se ha visto
fortalecida por el auge de la tecnología y las plataformas digitales
precisamente, un componente imprescindible que construye los cimientos
los cuales sustentarían a largo plazo esta dinámica relacional. Las
startups y emprendedores en el sector turístico han encontrado nuevas
formas de llegar a los viajeros, como el desarrollo de aplicaciones
móviles para reservar alojamiento, la creación de plataformas de
intercambio de experiencias locales o el uso de tecnologías emergentes
como la realidad virtual para mejorar la oferta turística. En los
últimos 20 años hemos sido testigos de algunas de estas propuestas como
lo es AirBnb. Estas innovaciones han transformado la manera en que los
turistas planifican y disfrutan sus viajes, mientras que simultáneamente
que ofrecen nuevas oportunidades de negocio para los emprendedores,
quienes dotan de una visión marcada superior que les permite identificar
nichos de mercado potencialmente aprovechables.

Así pues, el emprendimiento en el ámbito turístico ha permitido la
creación de soluciones innovadoras y personalizadas para los turistas
modernos, abriendo un mundo de posibilidades para aquellos emprendedores
dispuestos a tomar riesgos y desafiar las convenciones de la industria.
A medida que la tecnología continúa evolucionando y los viajeros
demandan experiencias más auténticas y personalizadas, la relación entre
el turismo y el emprendimiento seguirá siendo un campo fértil para la
innovación y la creación de valor. La información y los datos es una de
las herramientas más poderosas de la actualidad y que tiene una
aplicación enriquecida para maximizar el valor del turismo garantizando
su éxito a largo plazo.

\hypertarget{emprendimiento-en-la-actualidad}{%
\subsection{Emprendimiento en la
actualidad}\label{emprendimiento-en-la-actualidad}}

Primero sería conveniente darle una definición de concepto a lo que es
el emprendimiento. Al contrario que ocurre con otros conceptos teóricos,
en esta ocasión no hay tantas posibles definiciones que puedan
determinar el significado del emprendimiento como actos y ejemplos de
emprendimiento han podido haber, al menos y en esencia, en el contexto
turístico.

Las características son bastante similares en todos los casos ya que el
emprendimiento se puede traducir en tener la capacidad personal,
motivacional y material de llevar a cabo una serie de acciones para
sobresalir destacando por encima del resto aprovechando una oportunidad
de negocio que ha sido previamente visualizado y se le ha dado forma con
introspección mediante una idea en el cual se identifica una potencial
necesidad que puede ser cubierta con creativas soluciones. Y aunque la
palabra emprendimiento se le asocia mayormente a una persona particular
que pone en marcha un proyecto propio, no es estrictamente necesario que
deba tener ese requisito si atendemos a que los principios y la trivia
del emprendimiento pueden ser adoptados por una empresa ya existente con
el fin de querer innovar y diferenciarse de su competencia realizando
hazañas destacables por el que se le dé reconocimiento y éxito.

Esto puede crear cierta rigurosidad en cuanto a complejidad se refiere
si sabemos que el emprendimiento en la actualidad ha experimentado un
auge significativo, especialmente en el contexto del turismo, donde
valores como la resiliencia, la capacidad de ver y tener claro un
objetivo estratégico así como vinculado a todo ello los factores
cambiantes del entorno hacen más difícil una aplicación emprendedora en
un mercado globalizado a la vez que heterogéneo mientras que por ello
precisamente requiere que la propia filosofía emprendedora sea capaz de
darle una solución a este tipo de turismo tan rígido y arraigado. El
espíritu emprendedor ha encontrado un terreno fértil en esta industria,
donde los agentes emprendedores tienen la oportunidad de crear y ofrecer
soluciones innovadoras a los viajeros más contemporáneos. Por eso, el
acto de emprender en el turismo se centra en identificar y aprovechar
las necesidades no satisfechas de los turistas, así como en encontrar
formas creativas de mejorar la experiencia de viaje.

En primer lugar, los emprendedores en el sector turístico se dedican a
la creación de nuevas empresas, lo que podríamos considerar como
startups que buscan revolucionar la forma en que se concibe y se
disfruta el turismo, puede ser de manera incremental como ocurre en la
mayoría de los casos de intentos de innovación. Estos emprendedores
identifican nichos de mercado, investigan las tendencias emergentes y
desarrollan propuestas de valor únicas para atraer a los viajeros.
Pueden ofrecer desde servicios de alojamiento alternativos, como el
alquiler de viviendas particulares a través de plataformas digitales,
hasta actividades turísticas innovadoras y experiencias personalizadas
que se alejan de los circuitos turísticos tradicionales. En España un
ejemplo al que podríamos clasificar como relativamente innovadora es la
de poner en marcha rutas turísticas temáticas, que ha experimentado un
acelerado reconocimiento, como la Ruta del Vino de la Mancha que se hace
en Castilla-La Mancha aprovechando la figura popular de Don Quijote.

El emprendimiento en el turismo también se ha visto impulsado por el uso
de la tecnología y la digitalización, como se había mencionado en el
apartado anterior. Los emprendedores han sabido aprovechar el potencial
de las aplicaciones móviles, las plataformas en línea y las redes
sociales para llegar de manera más efectiva a los viajeros y promocionar
sus productos y servicios, ofreciendo soluciones y maneras de
experimentar exclusivas y algunas muy atrevidas. Por ejemplo, han
surgido plataformas dedicados exclusivamente a la experiencia autóctona
y auténtica en destino, como GetYourGuide, que han revolucionado la
forma en que las personas encuentran opciones de actividades en los
destinos a los que van de viaje. Asimismo, las empresas turísticas han
adoptado tecnologías como la realidad virtual y la inteligencia
artificial para mejorar la experiencia del viajero y ofrecer contenido
más inmersivo y personalizado. Por ejemplo, la Casa Milá de Barcelona,
un importante atractivo turístico ha apostado por la incorporación de
unas gafas de realidad mixta con el fin de reconstruir interactivamente
obras y trabajos de Antoni Gaudí en el interior de dicho edificio, con
el cual los viajeros pueden experimentar de primera mano.

Siguiendo con esta premisa, también se ha centrado en la sostenibilidad
y el turismo responsable. Cada vez más viajeros buscan experiencias que
sean respetuosas con el medio ambiente y las comunidades locales. Los
emprendedores han respondido a esta demanda creando empresas y servicios
que promueven la preservación del medio ambiente, la protección del
patrimonio cultural y la contribución positiva a las economías locales.
Esto es un interesante enfoque porque estaría en parte, dando por hecho
que una empresa emergente de tipo startup se forjase sobre directamente
sobre los principios de la responsabilidad social corporativa. Es la
razón por la que han aparecido empresas que ofrecen tours y actividades
turísticas que respetan la biodiversidad, utilizan energías renovables y
apoyan el desarrollo local, que pueden tener una potencialidad alta en
regiones rurales y no tanto urbanos.

Un aspecto clave del emprendimiento en el turismo es la capacidad de
adaptarse a las tendencias y cambios en el comportamiento del
consumidor. Los emprendedores deben estar atentos a las preferencias y
expectativas cambiantes de los viajeros, como la búsqueda de
experiencias auténticas, la personalización, la inmediatez y la
conveniencia. La flexibilidad y la capacidad de innovar rápidamente son
cruciales para tener éxito en este entorno competitivo que está varado
de viajeros que serán exigentes y premeditados en todo aquellos que
hagan y tendrán la destreza de estar atentos a las virtudes de lo nuevo
que le pueda proporcionar los destinos.

Además, el emprendimiento en el turismo implica enfrentar una serie de
desafíos y riesgos. Los emprendedores deben superar barreras como la
competencia, la falta de financiamiento, la regulación gubernamental, la
gestión de la incertidumbre y la volatilidad del mercado. También deben
ser capaces de gestionar y optimizar recursos como el capital, la
organización interna y la manera de superar obstáculos que irán
apareciendo durante y posterior a todo el proceso de ensamblaje de la
nueva empresa.

El emprendimiento sobre todo si es aplicado en el turismo se convierte
en un campo emocionante y desafiante que requiere creatividad,
motivación, adaptabilidad, predicción, previsión anticipatoria y visión
de futuro. En esta industria los usuarios emprendedores tienen la clave
de influir y transformar la forma en que los viajeros experimentan el
mundo, a través de la creación de empresas y servicios innovadores que
satisfacen las necesidades y deseos cambiantes poco sólidos. No sólo
impulsa la economía, sino que también contribuye al desarrollo
sostenible y la preservación cultural, promoviendo así un turismo más
responsable y enriquecedor para todos, que es un valor clave hoy en día
respaldado fuertemente por la ONU tal como lo recoge en sus Objetivos de
Desarrollo Sostenible.

Las bases para plantear un emprendimiento en un entorno turístico
volátil quedan dictaminadas robustamente por la puesta en marcha de una
infraestructura digital y organizacional que correlaciona los factores
turísticos más dinámicos e impredecibles con los criterios de iniciación
de un negocio. No es lo mismo tener la idea de comenzar en un negocio de
consultoría turística que ofrecer por poner un ejemplo de estilo dispar,
ofrecer servicios de traducción en destino. Tener en cuenta que el
avance y la masificada adopción de la web 3.0 en la actualidad
condiciona enormemente las herramientas con las que se puede dar forma a
distintos modelos de negocio y para poder llevar a cabo adecuadamente
distintas estrategias y tácticas de captación y consolidación de un
mercado esta ha de valerse de un recurso fundamental imposible de
ignorar: los datos.

\hypertarget{el-poder-de-los-datos}{%
\section{El poder de los datos}\label{el-poder-de-los-datos}}

Los datos son la representación simbólica de información cualitativa o
cuantitativa, es por ello por lo que se puede decir que es un medio por
el que se hace posible dependiendo de su construcción, elaboración y
desarrollo sistemático el permitir construir una serie de criptografía
de información apta para ser tratado y enfocada a una finalidad
determinada. Propiamente por sí solos, los datos no tienen un valor
significativo, pero cuando se combinan con observación y experiencia,
pueden ser útiles en investigaciones y hechos. En informática, el campo
de investigación donde más se hace uso de los datos, estos son
utilizados y manipulados para resolver problemas e incógnitas, crear
algoritmos autosuficientes y son expresiones que indican las cualidades
de los comandos en programación de toda clase. Además de ello, los datos
pueden ser símbolos numéricos utilizados en estudios para realizar
cálculos matemáticos y obtener información específica. También en
esencia, existe la categoría de datos personales, que se refiere a la
información contenida en la ficha o documento de identificación de la
persona física donde se ve una serie de datos descritos de manera mucho
más legible y cualitativa.

El Big Data, una increíble tecnología de inmensurable alcance y riqueza,
es la capacidad de procesar y analizar grandes volúmenes de datos de
diversas fuentes para extraer información y conocimientos valiosos, lo
que resalta el valor intrínseco de los datos en términos de su potencial
para generar ideas, tomar decisiones informadas, identificar patrones y
tendencias, y mejorar la eficiencia y efectividad de las organizaciones,
así como optimizar ciertos procesos complejos. El valor de estos datos,
transmitidos y tratados desde y hasta el Big Data radica en su capacidad
para proporcionar información relevante y oportuna que puede impulsar la
toma de decisiones estratégicas, mejorar la calidad de los productos y
servicios, optimizar las operaciones comerciales, personalizar la
experiencia del cliente, detectar fraudes, predecir comportamientos y
tendencias futuras, y permitir la innovación y el crecimiento en
diversas industrias y sectores. Lo que se consigue es amplificar y
potenciar el poder de los datos al proporcionar la infraestructura y las
herramientas necesarias para extraer conocimientos significativos y
transformarlos en ventajas competitivas y mejoras tangibles en diversos
aspectos empresariales y sociales, que se verían traducidos en distintos
beneficios y rendimientos económicos que pueden ser maximizados para ser
sostenidos a lo largo del tiempo.

Es importante para aprovechar las virtudes de la información en las
organizaciones. Es una forma de técnica el cual permite recuperar y
analizar grandes volúmenes de datos estructurados y no estructurados,
que no pueden ser procesados eficientemente con métodos tradicionales
anteriores al menos al web 2.0. El reto radica en el crecimiento rápido
de la información por analizar y de esto se observa que es necesario
correlacionarlos de manera eficiente para intentar obtener un resultado
de utilidad. El valor de los datos radica en su capacidad para
identificar tendencias, patrones y comportamientos de personas y
clientes por poner ejemplos anteriormente mencionados, lo cual es
esencial para mejorar los productos y servicios de una empresa
turística.

El poder de los datos con todo lo dicho es, en el marco teórico
turístico con el emprendimiento como enganche de accesorio dota de un
valor sin límites. Es un poder con el que por enumerar propiedades hace
posible generar ventajas competitivas a la hora de poner en marcha un
nuevo negocio turístico con un claro formato centrado en la tecnología a
través del cual se sintetiza como el mencionado Big Data o también con
la tendencia de los últimos dos años, la inteligencia artificial.
Valerse de herramientas adecuadas para gestionar una gran cantidad de
información traducido en un gran número de flujo turístico en un país
como España, es algo básico hoy en día, donde la contribución del
turismo al PIB oscila sobre el 10\% en estadísticas proporcionadas por
la OCDE en 2017.

Las características más notorias de tener unos buenos datos en el ámbito
turístico son los que hacen depender si un proyecto está destinado al
fracaso prematuro o a un éxito rotundo. Uno de los pasos a posteriori
que más se debería tener en cuenta es tener a mano una buena capacidad
de toma de decisiones que son elaboradas a partir del estudio y análisis
de un conjunto y serie de datos. La precisión y la predicción son
factores condicionantes anteriores a las acciones, mientras que hacer
una muestra verificada, con una adecuada segmentación y cruce de datos
contrastando resultados es crucial, como lo es también cruzar y cotejar
fuentes para asegurar que los datos son fiables y válidos de modo que no
se puedan ver fácilmente alterados tras su manipulación, estos permiten
tener una disposición de visión 360 sobre ella. A modo de recopilación,
los datos poseen un poder y valor significativo, ya que permiten obtener
información relevante para la toma de decisiones y generar ventajas
competitivas en el ámbito empresarial y no sólo turístico. En general,
se utilizan para identificar patrones, tendencias y comportamientos,
mejorar la calidad de productos y servicios, optimizar operaciones y
personalizar experiencias. En el campo turístico concretamente, pueden
emplearse para analizar preferencias de viajeros, personalizar ofertas,
mejorar la gestión de destinos, anticipar demandas, realizar análisis de
mercado, detectar oportunidades de negocio y promover la sostenibilidad
por mencionar sólo alguno de sus utilidades indiscriminadas. El uso
estratégico de datos en el turismo puede potenciar la innovación, la
eficiencia y la satisfacción del cliente, impulsando el crecimiento y la
competitividad en el sector, a la vez que fomenta la participación de
cada vez más usuarios dispuestos a dar de sí para contribuir a un
proceso emprendedor personal o colectivo que trae distintos tipos de
beneficios económicos, sociales o sostenibles.

\hypertarget{diferencia-entre-datos-e-informaciuxf3n}{%
\subsection{Diferencia entre datos e
información}\label{diferencia-entre-datos-e-informaciuxf3n}}

Como es evidente, se ha hablado numerosas veces de datos e información,
pero no es necesariamente un sinónimo de conceptos en el sentido
estricto y gramatical de la palabra para el uso que le queremos dar. Por
un lado, mientras que los datos es una representación simbólica y visual
de propiedades cuantitativas o cualitativas utilizadas posteriormente en
un mínimo pero elaborado proceso metodológico para lograr obtener un
determinado uso de valor amplificado, o dicho de otro modo, que son
primordialmente una clase de conocimientos, hechos o saberes brutos sin
contextualización los cuales carecen de significado por sí solos; no
obstante, por otro lado, la información a diferencia de esta se le puede
definir como precisamente el resultado de gestionar, darle tratamiento y
trabajar sobre dichos datos otorgándole un significado más estructurado
en el que conjuntamente su paso por el tratamiento habría permitido dar
una comprensión más profunda y útil del resultado obtenido y con la cual
se aplicaría a la resolución requerida sobre un cierto problema o
situación de relevancia.

Es interesante señalar esta pequeña diferencia ya que en la vida
cotidiana para desarrollar situaciones y resolver problemas pocos densos
y que no requieren alta cualificación se suele utilizar los datos y la
información como significados iguales para algunas cosas de considerable
redundancia. Por tanto, un dato es una representación que no describe
ninguna situación en concreta ni transmite ningún mensaje y su adecuada
utilización se traduce a lo que se llamaría información, que no es más
que el resultado de haber hecho uso y haber procesado ese conjunto de
datos con la finalidad de maximizar el rendimiento de información.

Cabe destacar que entonces, en un caso más práctico y en el contexto
turístico; no es lo mismo visualizar una gráfica estadística de llegadas
turísticas en un periodo de año debido a que como tal y por sí misma la
gráfica no proporciona nada, que por el contrario, analizar los datos
numéricos de esa representación estadística al ser comparado con otras
gráficas para obtener información precisa como entender la situación de
evolución de llegadas, que pueda ser a su vez absorbido por la empresa
en beneficio propio para distintas finalidades. Por ejemplo: en Abril de
2022 según la información de la INE en España hubo una llegada de 6,1
millones de turistas internacionales y simplemente con leer esa cifra de
ella no se saca ningún significado si no se compara con el número de
llegadas de Julio de 2022 que fue de 9 millones de turistas.

\hypertarget{emprender-en-turismo-con-datos}{%
\section{Emprender en turismo con
datos}\label{emprender-en-turismo-con-datos}}

En los siguientes capítulos se habla más en detalle de las
características más relevantes de este tema, pero por lo general, es
interesante tener conciso y claro algunos extractos que pueden ser
útiles a la hora de decidir cómo y de qué modo se puede iniciar un
proceso de emprendimiento en la industria turística valiéndose de los
datos como principal forma de penetración, posicionamiento y
diferenciación en el mercado.

Se ondea apartados de relevancia como las tecnologías y los métodos que
se ven implicados e involucrados en todo este complejo proceso de
emprendimiento, a continuación; pasando por cuestiones de importancia
como la ética hasta por factores de interés de negocio como asuntos
económicos, financieros y de mercado abarcando distintas modalidades que
directa o indirectamente influyen y afectan al desarrollo de un plan de
emprendimiento basado en datos.

Una de las palabras claves que pueden definir este panorama, fruto del
entorno cambiante y envuelto en la tecnología, es la innovación. El buen
uso de datos para obtener información valiosa con el que posteriormente
se procederá a emprender es una ventaja indiscutible en un mundo donde
la percepción de lo mismo es algo bastante habitual a ojos de un viajero
más de estilo aspiracional que racional.

El uso de datos como algo de relevancia especialmente si se tiene en
cuenta el auge de la ciencia de datos es actualmente en perspectiva un
nicho de mercado, pero es una tendencia creciente en las industrias
globalizadas que aprovecha la información generada para lograr tener
innovación y mejorar la experiencia del turista. Al combinar el espíritu
emprendedor con el análisis de datos, los emprendedores pueden
identificar oportunidades, tomar decisiones informadas y crear
soluciones personalizadas para satisfacer necesidades imprevistas. Al
margen de este desarrollo, por poner un caso de éxito, en España se está
desarrollando los llamados Destinos Turísticos Inteligentes, impulsados
por Segittur en formato de red unificada donde la base fundamental de su
construcción se basa en los datos para conseguir una propuesta
tecnológica innovadora en aquellos destinos seleccionados y aptos para
esta transformación digital y tecnológica. Así, San Sebastián según este
organismo público está clasificado como un Destino Turístico
Inteligente.

El proceso emprendedor en turismo puede utilizar datos para comprender
mejor las preferencias y comportamientos de los viajeros, ampliando
horizontes de entendimiento introspectivo. Mediante el análisis de datos
demográficos, de comportamiento de reserva y de opiniones en redes
sociales, los emprendedores pueden identificar patrones y tendencias que
les permiten adaptar sus ofertas y servicios a las demandas cambiantes
del mercado, previo a los pasos de identificación de necesidades.

Además, los datos son fundamentales para personalizar la experiencia del
cliente, que es un rasgo cada vez más común donde la especialización de
necesidades es algo que es requerido en cada vez más segmentos de
viajeros, lo general no llama la atención en el presente. Se utilizan
datos recopilados durante el proceso de reserva y estancia para ofrecer
recomendaciones personalizadas, sugerir actividades y mejorar la
satisfacción general del viajero. Esto lo que es capaz de hacer es por
un lado ofrecer una experiencia más individualizada y por otro, aumenta
la fidelidad del cliente final.

El análisis de datos como disciplina de la ciencia de datos, también
puede ayudar a los emprendedores en turismo a optimizar sus operaciones
internas. Mediante el seguimiento de indicadores clave de rendimiento,
como las tasas de ocupación, los costos operativos y las preferencias de
los clientes proporcionado por bases de datos estadísticos como UNWTO,
WTTC o INE, los emprendedores principiantes pueden tomar decisiones más
informadas sobre la gestión de recursos, la fijación de precios y la
asignación de personal, si es que lo necesitan en la formalización de
una estructura empresarial.

Otra área donde los datos son valiosos es en la identificación de
oportunidades de mercado. Pueden utilizarlos para analizar la
competencia, identificar nichos no explotados y desarrollar estrategias
de diferenciación ayudados por técnicas de análisis propios de un ámbito
empresarial como el DAFO o las 5 Fuerzas de Porter sobre los que se
refleja la consiguiente información analizada. Esto les permite ofrecer
productos y servicios únicos que se destaquen en el mercado y atraigan a
nuevos segmentos de clientes.

El emprendimiento en turismo también se beneficia de la capacidad de los
datos para predecir tendencias y comportamientos futuros, es una especie
de criterio de análisis que se hace cuando se realiza predicción de
futuro con bases de datos absolutamente descomunales propios del Big
Data. Mediante el análisis de datos históricos y el uso de técnicas de
pronóstico, predicción como más concretamente el de regresión, los
usuarios emprendedores pueden anticipar la demanda, identificar
temporadas altas y bajas que son muy útiles para abarcar ocupaciones de
habitaciones hoteleras, y ajustar sus estrategias de marketing y
operativas en consecuencia.

Además, los datos son esenciales para la toma de decisiones
estratégicas. Los emprendedores en turismo pueden utilizar análisis de
datos para evaluar la viabilidad de nuevos proyectos, medir el retorno
de la inversión y evaluar el impacto de sus decisiones calculando una
probabilidad de certeza de éxito. Esto les permite minimizar los riesgos
y maximizar las oportunidades de éxito.

Con lo cual, el emprendimiento en turismo utilizando datos implica el
uso estratégico de información detallada para identificar oportunidades,
ser diferenciadores, saber innovar, optimizar operaciones, identificar
nichos de mercado, predecir tendencias, tomar decisiones informadas,
adoptar posiciones estratégicas y maximizar el éxito empresarial. Los
datos son una herramienta poderosa como habríamos explicado en el
epígrafe de su poder con el cual permite a los emprendedores en un
mercado tan volátil como el turístico, crear soluciones de vanguardia y
cubrir las necesidades cambiantes de sus clientes, impulsando así el
crecimiento y la competitividad en la industria pero también atendiendo
a algunos valores modernos, impulsar la participación colaborativa de
los grupos de interés y fomentar un correcto desarrollo de relaciones
beneficiosas para todos, siempre por supuesto poniendo el foco que es un
negocio que busca obtener beneficios para el emprendedor.

\hypertarget{referencias}{%
\section{Referencias}\label{referencias}}

Adrián, Y. (2015, April 18). Datos. Concepto de - Definición de;
ConceptoDefinicion.de. https://conceptodefinicion.de/datos/ Cardenas, F.
(2023, March 22). Qué es un emprendimiento, características y ejemplos
exitosos. Hubspot.es. https://blog.hubspot.es/sales/guia-emprendimiento
/cronologia/-/meta/redaccion-e-n.~(2016, July 7). Diferencia entre dato,
información y conocimiento. Estrategia y Negocios.
https://www.estrategiaynegocios.net/centroamericaymundo/diferencia-entre-dato-informacion-y-conocimiento-HFEN977752
Emprender en turismo, conceptos clave para iniciar un negocio. (2020,
June 4). Ostelea.com.
https://www.ostelea.com/actualidad/blog-turismo/tendencias-en-turismo/emprender-en-turismo-conceptos-clave-para-iniciar-un-negocio
Glosario de términos de turismo. (n.d.). Unwto.org. Retrieved June 15,
2023, from https://www.unwto.org/es/glosario-terminos-turisticos
Herrera, F. (2016, August 2). El poder de los datos: aplicaciones del
Big Data. Fundación Aquae.
https://www.fundacionaquae.org/el-poder-de-los-datos/ Hiberus. (2021,
August 19). Beneficios del Big Data para el sector turístico. Blog de
Hiberus Tecnología.
https://www.hiberus.com/crecemos-contigo/beneficios-del-big-data-para-el-sector-turistico/
Hosteltur. (2023, June 7). Gestión de datos en turismo, una oportunidad
para Europa. Hosteltur.
https://www.hosteltur.com/157717\_gestion-de-datos-en-turismo-una-oportunidad-para-europa.html
Noguera, B. (2014, November 20). Cuál es la diferencia entre dato e
información. Culturación.
https://culturacion.com/cual-es-la-diferencia-entre-dato-e-informacion/
Número de turistas según país de residencia. (n.d.). INE. Retrieved June
15, 2023, from https://www.ine.es/jaxiT3/Datos.htm?t=10822 Red de
Destinos Turísticos Inteligentes en España. (2019, February 18).
SEGITTUR.
https://www.segittur.es/destinos-turisticos-inteligentes/proyectos-destinos/red-dti/
Taboada, J. (2017, December 1). Big data El poder de los datos. TYS
Magazine. https://tysmagazine.com/big-data-poder-los-datos/

\bookmarksetup{startatroot}

\hypertarget{muxe9todos-y-tecnologuxedas}{%
\chapter{2. Métodos y Tecnologías}\label{muxe9todos-y-tecnologuxedas}}

Autora: Ana Gabriela Echeverría Solís

\hypertarget{objetivos-del-capuxedtulo}{%
\section{2.1 Objetivos del capítulo}\label{objetivos-del-capuxedtulo}}

Orientar al emprendedor en el ámbito turístico sobre los métodos
aplicables a la toma de decisiones basada en datos, proporcionando una
descripción detallada y pasos específicos para lograrlo, así como, la
tecnología necesaria desde la recopilación de datos, hasta la aplicación
práctica.

\hypertarget{introducciuxf3n-al-capuxedtulo.}{%
\section{2.2 Introducción al
capítulo.}\label{introducciuxf3n-al-capuxedtulo.}}

\begin{quote}
\begin{quote}
Los datos no son información, la información no es conocimiento, el
conocimiento no es comprensión, la comprensión no es sabiduría'',
Cliffor Paul Stoll, astrofísico estadounidense (2006).
\end{quote}
\end{quote}

Esta frase resalta la importancia de ir más allá de la simple
recopilación de datos y aprovecharlos de manera efectiva para obtener
conocimiento y sabiduría. En la vida empresarial, es común acumular
datos sobre transacciones y operaciones diarias. En el ámbito del
turismo, el enfoque se amplía aún más al aprovechar el big data en
tiempo real, que proporciona información valiosa sobre los viajeros, sus
trayectos, preferencias y prioridades. Este vasto conjunto de datos
ofrece amplias oportunidades para optimizar las operaciones de viaje,
personalizar las ofertas, mejorar la prestación de servicios y abrir
nuevos canales de negocio. En este capítulo, se explorarán los métodos y
herramientas disponibles para aprovechar al máximo los datos en el
emprendimiento turístico. Y se presentará una ruta para convertir los
datos en información que pueda ser utilizada para impulsar el éxito en
el mundo del turismo.

Antes de comenzar a profundizar en tópico de este capítulo, será
importante recordar que, la recabación de datos ha existido desde hace
décadas, sin embargo, el factor diferencial que distingue a estos
tiempos es el grado de madurez que se da entre, la cantidad de datos que
se han generado, y acumulado, procedentes de comentarios en blogs,
correos electrónicos, libros y artículos digitales, consultas online,
transacciones económicas electrónicas, entre otros; conjugado a la
tecnología y las herramientas digitales, que tienen la capacidad de
procesar grandes volúmenes, de variadas tipologías, a una altísima
velocidad y, la capacidad de discernir de que esos datos sean ciertos o
veraces, en grandes razgos, eso es Big Data. Según la empresa consultora
IBM, el término inglés Big Data hace referencia a todos aquellos
conjuntos de datos cuyo tamaño supera la capacidad de búsqueda, captura,
almacenamiento, gestión, análisis, transferencia, visualización o
protección legal de las herramientas informáticas convencionales. En
esencia, el Big Data posibilita el estudio y explotación inteligente de
millones de bytes de información sobre toda clase de fenómenos y
actividades producida, difundida o almacenada a través de teléfonos
móviles, redes sociales o, por ejemplo, máquinas conectadas al internet
de las cosas (IoT).

Erróneamente, algunos emprendedores piensan que, iniciar en el mundo del
Big Data es sólo para las empresas grandes, otros se entusiasman con la
idea y, sin una buena planeación y metodología, comienzan a acumular
datos de diversas fuentes con la esperanza de que puedan ser de utilidad
para el futuro. Sin embargo, aún cuando los costos de almacenamiento
pueden no ser significativos, recabar datos, filtrarlos, analizarlos y
conservarlos requieren recursos, dinero y tiempo.

Así que, antes de iniciar con una estrategia Big Data, es importante
estar al tanto de cuáles son las tendencias actuales del mercado, las
aplicaciones que se están dando, así como, las políticas de seguridad en
cuanto a la protección de la privacidad y las legislaciones de cada
país. Lo anterior permitirá poder hacer una buena planeación de la
arquitectura digital que se necesita para el proyecto a emprender, pues
aún cuando no se tenga claro si se está en disposición de poder abordar
una estrategia Big Data, estará listo al momento de detectar una
oportunidad. La evidencia demuestra que los casos de éxito vienen
acompañados de, a) un liderazgo capaz de definir los objetivos,
desarrollar e implementar las estrategias, métodos y tecnologías que la
organización requiere; b) una cultura organizacional digitalizada
orientada a la recabación de datos y a la toma de decisiones basada en
ellos; y c) una infraestructura tecnológica adecuada a las necesidades
del proyecto, que permita la escalabilidad, interoperabilidad,
accesibilidad, almacenamiento y seguridad de la información.

\hypertarget{por-duxf3nde-comenzar-un-proyecto-de-emprendimiento-turuxedstico-o-la-modificaciuxf3n-de-una-empresa-turuxedstica-en-curso-hacia-una-cultura-orientada-a-la-toma-de-decisiones-basada-en-datos.}{%
\section{2.3 Por dónde comenzar un proyecto de emprendimiento turístico
o la modificación de una empresa turística en curso, hacia una cultura
orientada a la toma de decisiones basada en
datos.}\label{por-duxf3nde-comenzar-un-proyecto-de-emprendimiento-turuxedstico-o-la-modificaciuxf3n-de-una-empresa-turuxedstica-en-curso-hacia-una-cultura-orientada-a-la-toma-de-decisiones-basada-en-datos.}}

\hypertarget{cinco-pasos-para-fomentar-la-adopciuxf3n-de-big-data-en-los-proyectos-de-emprendimiento-turuxedstico-basados-en-el-anuxe1lisis-realizado-por-ibm-de-las-conclusiones-del-estudio-big-data-wirk-study.}{%
\subsection{2.3.1 Cinco pasos para fomentar la adopción de Big Data en
los proyectos de emprendimiento turístico, basados en el análisis
realizado por IBM de las conclusiones del estudio ``Big Data @ Wirk
Study''.}\label{cinco-pasos-para-fomentar-la-adopciuxf3n-de-big-data-en-los-proyectos-de-emprendimiento-turuxedstico-basados-en-el-anuxe1lisis-realizado-por-ibm-de-las-conclusiones-del-estudio-big-data-wirk-study.}}

\textbf{a). Dedicar los esfuerzos iniciales a resultados centrados en el
cliente.} Es esencial que las empresas turísticas enfoquen sus esfuerzos
de Big Data en áreas que generen el máximo valor para el negocio. Esto
implica comenzar por analizar a los clientes para brindarles un mejor
servicio al comprender sus necesidades y anticiparse a sus
comportamientos futuros. La digitalización masiva ha cambiado el
equilibrio de poder entre los individuos y las instituciones, por lo que
las empresas deben enfocarse en conocer a sus clientes como individuos y
emplear nuevas tecnologías y análisis avanzado para comprender mejor sus
interacciones y preferencias. Sin embargo, para cultivar relaciones
valiosas, las empresas deben conectarse con los clientes de formas que
ellos perciban como valiosas, ya sea a través de interacciones más
oportunas, informadas o relevantes, o mejorando las operaciones
subyacentes para mejorar la experiencia general. \textbf{b). Desarrollar
un proyecto de Big Data para toda la empresa.} Esto es esencial para
alinear las necesidades de los usuarios de negocio con la implementación
de TI y establecer una visión, estrategia y requisitos claros.
\textbf{c). Comenzar con los datos existentes para lograr resultados a
corto plazo.} El lugar más lógico (y rentable) para comenzar a buscar
nuevos conocimientos es dentro de la empresa. Para los proyectos que aún
no han iniciado, esto sería los datos publicados de fuentes oficiales
dentro del área o territorio en el que se pretende incursionar. Buscar
internamente, le permitirá al emprendedor una visión mejorada de los
requerimientos en cuanto a la estructura de los datos, software,
habilidades, que más adelante le permitirán gestionar mayores volúmenes
y variedades de datos. \textbf{d). Desarrollar funcionalidades
analíticas sobre la base de prioridades de negocio.} Una prioridad para
el emprendedor turístico será fortalecer sus habilidades analíticas,
funcionales y de TI. De igual importancia, centrar la atención en el
desarrollo profesional del equipo y el avance de la trayectoria de los
analistas internos, que se familiaricen con los retos y procesos del
proyecto o negocio. \textbf{e). Diseñar la empresa sobre la base de
resultados cuantificables.} Las soluciones de big data más eficaces
identifican primero los requisitos del proyecto o negocio y, después,
adaptan la infraestructura, las fuentes de datos y el análisis
cuantitativo.

\hypertarget{cuatro-fases-a-las-que-se-enfrenta-un-emprendedor-durante-la-adopciuxf3n-de-big-data.-ibm---oxford-university-2012}{%
\subsection{2.3.2 Cuatro fases a las que se enfrenta un emprendedor
durante la adopción de Big Data. (IBM - Oxford University,
2012)}\label{cuatro-fases-a-las-que-se-enfrenta-un-emprendedor-durante-la-adopciuxf3n-de-big-data.-ibm---oxford-university-2012}}

\textbf{a). Educar para explorar.} El emprendedor y su equipo deberán
centrarse en ampliar permanentemente sus conocimientos, centrándose en
los que les ofrecerán una ventaja competitiva. Identificar las
oportunidades y retos, dentro del proyecto o negocio, que se pueden
abordar mejor con un acceso oportuno a la información. \textbf{b).
Explorar para interactuar.} Como requisito, el emprendedor deberá
mantener su compromiso a lo largo del desarrollo de la estrategia, así
como, un respaldo activo a los miembros del equipo que participen en
ella. Deberán desarrollar la ruta crítica a seguir definiendo los datos,
su estructura, la tecnología y las habilidades que se requieren, a corto
y a largo plazo. Establecer por dónde comenzar y cómo desarrollar el
plan en consonancia con la estrategia de negocio de la empresa.
\textbf{c). Interactuar para ejecutar.} En esta fase se comienzan a
comprobar los beneficios para el proyecto o negocio derivados del Big
Data. Los éxitos se deben promover activamente. También es importante
las reuniones de retroalimentación y evaluación para identificar las
modificaciones y mejoras que se deben incorporar. Un consejo muy valioso
en esta fase, y en la vida empresarial general, es documentar los
hallazgos y las justificaciones bajo una metodología descriptiva. Esto
trae múltiples beneficios, uno de ellos es la posibilidad de replicar o
migrar proyectos piloto a otras áeras de la empresa que reúnen
requisitos similares. \textbf{d). Ejecutar: aceptar la innovación de big
data.} Documentar los resultados cuantificables de los primeros éxitos
para reforzar futuras iniciativas, poner en marcha comunicaciones
formales sobre big data para aumentar y fomentar la cultura empresarial
orientada al dato, centrarse en la evaluación sobre la tecnología y
habilidades necesarias para los retos en big data por venir.

\hypertarget{tuxe9cnicas-y-tecnologuxedas}{%
\section{2.4 Técnicas y
Tecnologías}\label{tuxe9cnicas-y-tecnologuxedas}}

En el contexto del emprendimiento en el ámbito de los datos, es crucial
adquirir conocimientos acerca de las tecnologías principales disponibles
en el mercado actual y las técnicas para realizar análisis de datos.
Para los emprendedores que poseen una gran determinación pero escasas
habilidades, resulta fundamental familiarizarse con las características
destacadas de las tecnologías más comunes, aunque cabe destacar que
existen numerosas tecnologías adicionales que evolucionan y surgen
constantemente. Asimismo, es importante conocer los nombres relevantes y
comenzar a explorar las técnicas de análisis, lo cual permitirá trabajar
con volúmenes y variedades de datos más reducidos inicialmente, pero con
el potencial de ampliar exitosamente hacia entornos más complejos en el
futuro. Esta etapa inicial de familiarización sienta las bases para un
progreso fructífero en el ámbito del análisis de datos. A continuación,
se presentan algunas de ellas extraídas del documento publicado por
McKinsey Global Institute, (2011) ``Big data: The next frontier for
innovation, competition, and productivity''.

\hypertarget{tuxe9cnicas-para-la-analuxedtica-big-data-utilizadas-frecuentemente-en-la-industria-turuxedstica.}{%
\subsection{2.4.1 Técnicas para la analítica Big Data utilizadas
frecuentemente en la industria
turística.}\label{tuxe9cnicas-para-la-analuxedtica-big-data-utilizadas-frecuentemente-en-la-industria-turuxedstica.}}

\textbf{Test A/B:} Técnica mediante la que un elemento o grupo control
``A'' se compara con diversos elementos de prueba o grupos de prueba
``B'' con el fin de determinar cuál de ellos o qué tratamiento sobre los
mismos supone una mejora para un determinado objetivo. \emph{Un caso de
uso muy extendido de esta técnica es para determinar qué textos,
imágenes, colores o disposición de los elementos, mejoran las tasas de
conversión en una página web.} \textbf{Reglas de asociación (Association
rule learning):} Permiten descubrir relaciones relevantes o reglas de
asociación entre variables de datos. Se basan en una gran variedad de
algoritmos que generan y prueban posibles reglas. Una de sus
aplicaciones es para el análisis de qué productos se compran con
frecuencia juntos. \emph{Un ejemplo de esto sucede al momento de
reservar un boleto de avión y los servicios complementarios como
traslados, hospedaje, seguros, entre otros.} \textbf{Clasificación:}
Utilizado en Data Mining. Conjunto de técnicas para identificar la
categoría a la que pertenece un nuevo conjunto de datos basándose en
clasificaciones realizadas con anterioridad. Estas técnicas se denominan
de aprendizaje supervisado porque parten de un conjunto de datos de
entrenamiento con conjuntos de datos ya clasificados. \emph{Se utilizan
por ejemplo en los sistemas de recomendación para ayudar al usuario a
planificar su viaje o lo que va a hacer en una ciudad. Otro caso de uso
muy frecuente es la predicción del comportamiento del cliente en las
decisiones de compra.} \textbf{Análisis de grupos (clustering / cluster
analysis):} Utilizado en Data Mining. Método estadístico para la
clasificación de objetos que se basa en dividir un grupo de elementos en
grupos más pequeños de objetos similares, cuyas características de
similitud no se conocen de antemano. \emph{Un ejemplo de utilización del
clustering es la segmentación de consumidores en grupos análogos para
realización de campañas de marketing concretas.} \textbf{Crowdsourcing:}
Técnica para recolectar datos de un gran grupo de personas o comunidad a
través de una invitación a participar, generalmente de una red
empresarial o social. \emph{Algunos hoteles han utilizado esta técnica
para involucrar a la comunidad en la planeación o diseño de servicios.}
\textbf{Fusión e integración de datos:} Conjunto de técnicas que
integran y analizan los datos de múltiples fuentes con el fin de
establecer planteamientos que sean más eficientes y potencialmente más
precisos que si se establecieran mediante el análisis de una única
fuente de datos. \emph{La geolocalización de un dispositivo sólo indica
dónde está pero si se combina con un mapa posibilita conocer su
ubicación. La información de las redes sociales, analizada por un
lenguaje de procesamiento natural, puede ser combinada en tiempo real
con los horarios de venta, y así determinar qué efecto tienen las
campañas de marketing en el comportamiento de compra y en sus
emociones.} \textbf{Data mining:} Combina métodos estadísticos y de
aprendizaje automático para extraer patrones de grandes conjuntos de
datos. Estas técnicas incluyen reglas de asociación, clustering,
clasificación y regresión. \emph{Se aplica por ejemplo para determinar
los segmentos de cliente con más probabilidad de responder a una oferta,
identificar las características de los empleados más exitosos o qué
patrones se repiten en el comportamiento de los clientes que compran con
mayor frecuencia cierto tour o reservan cierto tipo de habitaciones.}
\textbf{Algoritmos genéticos:} Utilizados para la optimización e
inspirados en el proceso de selección natural. Estos algoritmos son muy
adecuados para la solución de problemas no lineales. \emph{Ejemplos de
aplicaciones incluyen la mejora de la planificación de tareas en la
fabricación, la optimización del rendimiento de una cartera de
inversiones o la resolución del problema del viajero.}
\textbf{Aprendizaje automático o Machine learning:} Especialidad de la
inteligencia artificial que se ocupa del diseño y desarrollo de
algoritmos que permiten a los ordenadores aplicar ``inteligencia'' a
partir de datos empíricos. \emph{El objetivo principal es aprender a
reconocer de forma automática patrones complejos y tomar decisiones
inteligentes.} El análisis de sentimiento de los textos para
clasificarlos en positivos, negativos o neutros se realiza mediante
técnicas de aprendizaje automático. \textbf{Lenguaje de procesamiento
natural (NLP):} Conjunto de técnicas de inteligencia artificial y
lingüística para analizar el lenguaje humano. \emph{Una de sus
aplicaciones más extendidas es el los motores de búsqueda donde según se
está escribiendo se citó completa la palabra o se recomienda la palabra
siguiente, basándose en búsquedas anteriores y en secuencias de palabras
que aparecen juntas.} \textbf{Redes neuronales:} Son modelos
computacionales, inspirados en la estructura y funcionamiento de las
redes neuronales biológicas con el objetivo de encontrar patrones en los
datos. \emph{Se utilizan para la identificación de clientes de alto
valor que están en riesgo de causar baja o para detectar reclamaciones
de seguros fraudulentas.} \textbf{Análisis de redes:} Conjunto de
técnicas para caracterizar las relaciones entre nodos en un gráfico o
una red. \emph{Muy utilizado en el análisis de redes sociales para
determinar las concesiones entre los individuos de una comunidad, cómo
viaja la información, o quién tiene mayor influencia sobre quién.
También para identificar a los actores clave de opinión dentro del
mercado meta o para identificar cuellos de botella dentro de los flujos
de información.} \textbf{Optimización:} Técnicas numéricas utilizadas
para rediseñar sistemas y procesos complejos que mejoren sus cometidos
de acuerdo a una o más medidas objetivas como pueden ser el coste, la
velocidad o la fiabilidad. \emph{Ejemplos de aplicaciones incluyen la
mejora de proceso operativos tales como la programación,
direccionamiento o distribución en plantas.} \textbf{Modelos
predictivos:} Técnicas mediante las que se crea o elige un modelo
matemático para predecir la probabilidad de un resultado.* Un ejemplo de
aplicación es la predicción con antelación de la entrada de turistas
extranjeros y las pernoctaciones previstas. \emph{La técnica de
regresión es un ejemplo de este modelo.} \textbf{Regresiones:} Utilizado
en Data Mining. Modelos estadísticos para determinar cómo el valor de
una variable dependiente cambia cuando se modifican una o más variables
independientes. A menudo se utilizan para el pronóstico o la predicción.
\emph{Se pueden utilizar para la previsión de volúmenes de ventas en
base a diferentes variables económicas y mercados o para la
determinación de los parámetros de fabricación que más influyen en la
satisfacción del cliente.} \textbf{Análisis de sentimiento:} Aplicación
de lenguajes de procesamiento natural (NLPs por sus siglas en inglés) y
otras técnicas analíticas para identificar y extraer información
subjetiva de comentarios y textos. Los aspectos clave de estos análisis
incluyen la identificación de las características, aspectos, o productos
sobre los que se está expresando un sentimiento, y la determinación del
tipo de sentimiento (positivo, negativo o neutro) y el grado y la fuerza
del sentimiento. \emph{Un ejemplo de utilización muy extendido son las
herramientas de monitorización de blogs, páginas web y redes sociales
para determinar cómo los clientes y grupos de interés reaccionan a sus
acciones, opinan sobre sus productos.} \textbf{Análisis espaciales:}
Modelos que analizan las propiedades topológicas, geométricas, o
geográficas codificadas en un conjunto de datos. Estos datos suelen ser
generados por sistemas de información geográfica (GIS por sus siglas en
inglés) que proporcionan información de ubicación. \emph{Se usan para
determinar la predisposición del consumidor a comprar un producto en
función de su ubicación, o para la simulación de cómo incrementar la
eficiencia de cada cadena de producción ubicada en diferentes
localizaciones.} \textbf{Simulación:} Consiste en la modelización del
comportamiento de sistemas complejos, a menudo utilizados para la
previsión, predicción y planificación de escenarios. Existen algoritmos
que ejecutan miles de simulaciones basadas en diferentes supuestos con
muestras de datos aleatorias para la obtención de histogramas con las
distribuciones probabilísticas de los resultados. \emph{Un caso de uso
es la evaluación del cumplimiento de los objetivos financieros teniendo
en cuenta las incertidumbres sobre el éxito de diversas iniciativas.}
\textbf{Análisis de series temporales:} Conjunto de técnicas
estadísticas y de procesamiento de señales para el análisis de
secuencias de datos en momentos de tiempo correlativos para la
extracción de patrones y características significativas en los datos.
\emph{Se utilizan por ejemplo para encontrar patrones que pretenden a la
ocurrencia de un territorio.} \textbf{Visualisation:} Técnicas
utilizadas para crear imágenes, diagramas o animaciones, para comunicar,
entender y mejorar el resultado del análisis Big Data.

\hypertarget{tecnologuxedas}{%
\subsection{2.4.2 Tecnologías}\label{tecnologuxedas}}

\textbf{Big Table:} Sistema de gestión de base de datos distribuidos.
creado por Google. Inspiración para HBase. \textbf{Business intelligence
(BI):} Aplicación de software diseñado para elaborar reportes, analizar
y presentar los datos. Las herramientas BI son utilizadas frecuentemente
para leer datos que han sido previamente almacenados en Datawarehouse.
También son utilizadas para crear reportes periódicos o para mostrar
cuadros de mando en tiempo real. Ejemplos de estas herramientas son:
Power BI, Locker, Tablou, que utilizan la tecnología de visualización
para presentar la información de manera atractiva utilizando gráficas
tipo (tag cloud, clustegram, histogramas, gráficas espaciales, entre
otros) \textbf{Cassandra:} Sistema de gestión de base de datos open
source diseñado para manejar enormes cantidades de datos en un sistema
distribuido. Este sistema fue desarrollado originalmente por Facebook y
ahora está gestionado como un proyecto de la Apache Software Foundation.
\textbf{Cloud computing:} Modelo tecnológico en el que se proporciona un
servicio para el acceso bajo demanda a un conjunto de recursos
informáticos compartidos de forma flexible e instantánea. \textbf{Data
mart:} Tipo de Data Warehouse, utilizado para proveer datos a los
usuarios a través de las herramientas BI. \textbf{Dyanmo:} Sistema de
almacenamiento de datos desarrollado por Amazon. \textbf{ETL, extract,
transform and load / extraer, transformar, cargar:} Herramientas
utilizadas para extraer datos de fuentes externas, transformarlas para
que cumplan con los requerimientos y cargarlas en una base de datos o en
un Data Warehouse. \textbf{Sistemas de distribución / Distributed
System:} Están constituidos por varios equipos que se comunican en red y
que son utilizados para resolver conjuntamente un problema
computacional. El problema se divide en múltiples tareas, que se asignan
a uno o más ordenadores para ser resueltas en paralelo. El beneficio de
los sistemas distribuidos es que ofrecen un mayor rendimiento a un coste
menor, mayor fiabilidad y más escalabilidad. \textbf{Hadoop:} Software
open sonríe para el procesamiento de grandes conjuntos de datos en un
sistema distribuido. Su desarrollo fue inspirado por el Sistema de
Archivo de Google y Google MapReduce. Originalmente fue desarrollado en
Yahoo! Y ahora está gestionado como un proyecto de la Apache Software
Foundation. \textbf{HBase:} Sistema de base de datos no relacional,
distribuido y open sobre, que se basa en el sistema Bigtable de Google.
Fue desarrollado originalmente por Powerset y ahora se gestiona como un
proyecto de la fundación Apache Software como parte de Hadoop.
\textbf{MapReduce:} Modelo de programación utilizado por Google, para
procesar enormes conjuntos de datos que se emplea para la resolución de
algunos algoritmos susceptibles de ser párale Liza les y procesados en
sistemas distribuidos. Implementado en Hadoop. \textbf{Mashup:}
Aplicación que utiliza y combina los datos de una o más fuentes para
crear nuevos servicios. \textbf{Metadata:} Datos que describen el
contenido o el contexto de archivos de datos. \textbf{MongoDB:} Sistema
de gestión de base de datos open source diseñado para trabajar con datos
no estructurados, haciendo que la integración de los datos en ciertas
aplicaciones sea más fácil y rápida. \textbf{R:} Lenguaje y entorno de
programación open source para el análisis estadístico y gráfico. Se ha
convertido en un estándar entre los estadísticos para el desarrollo de
software estadístico y es ampliamente utilizado para análisis de datos.
\textbf{Procesamiento de eventos complejos:} Tecnología diseñada para
detectar y responder a eventos en tiempo real que indican situaciones
que impactan sobre los negocios. Se utilizan para detección de fraude,
sistemas financieros, servicios basados en la localización.

\hypertarget{muxe9todo}{%
\section{2.5 Método}\label{muxe9todo}}

El proceso de toma de decisiones se vuelve más eficiente cuando se
incorporan herramientas y técnicas de análisis para el manejo de datos,
ya que facilitarán la interpretación de los mismos y la identificación
de tendencias, para explorar, explicar o predecir una situación
determinada. El análisis de datos, adquiere un nuevo nivel y perspectiva
al utilizarlo como fuente para la construcción de indicadores de gestión
o KPIs. Para ello, resulta evidente que es fundamental definir con
claridad qué indicadores se quiere generar, mismos que por supuesto
deberán estar alineados a los objetivos estratégicos de la organización.
Sin embargo, hay un paso esencial antes de definir los indicadores:
definir el objetivo del análisis. Ya sea la personalización de
servicios, la satisfacción del cliente, el monitoreo de un proceso, la
optimización de precios, analizar los comentarios y opiniones de los
clientes, optimizar la gestión del inventario, elaborar campañas de
marketing digital, o cualquier otro objetivo, el punto de partida en
común será definir con claridad qué se quiere obtener como resultado del
método de análisis.

La evidencia demuestra que seguir un método sistemático orienta al
emprendedor y le sirve de guía para facilitar la consecución de un
objetivo. A continuación se presenta el método sugerido para emprender
en datos, destacando que estos pasos no son necesariamente lineales y
pueden requerir iteraciones y ajustes a lo largo del proceso.

\begin{quote}
\begin{itemize}
\tightlist
\item
  \textbf{Definir el grupo de trabajo.} Ya sea que se vaya a emprender
  en solitario o en grupo, definir quiénes deberán participar en el
  desarrollo e implementación del método.
\item
  \textbf{Definición del objetivo de análisis.} Surge de la observación
  de una necesidad a resolver. Entendiendo que hay una discrepancia
  entre un estado actual y un estado deseado.
\item
  \textbf{Identificación de los criterios e indicadores} que serán
  relevantes para la medición de antes, durante y después del análisis.
  Pensemos en estos indicadores, como fotografías de lo que se está
  evaluando. Si es necesario, se deberán ponderar los criterios, ya que
  no todos son de la misma importancia. Según Castro (2013), los
  indicadores deben cumplir con ciertas características básicas:
  simplicidad (capacidad de definir el evento de manera clara, económica
  y oportuna), adecuación (capacidad para reflejar la magnitud del hecho
  analizado), validez en el tiempo (ser permanente por un período de
  tiempo determinado), participación de los usuarios (involucrar al
  equipo que trabajará con el indicador desde su diseño), oportunidad
  (capacidad para que los datos sean recolectados en el momento justo).
\item
  \textbf{Investigación} de fuentes primarias y secundarias,
  cerciorándose de que las fuentes sean confiables y estén actualizadas.
  Elaborar un listado e identificar las fuentes de datos necesarias para
  respaldar el test de viabilidad. Pueden ser datos demográficos,
  estadísticas turísticas, informes de la industria, estudios de
  mercado, datos de competidores, entre otros, todo depende del
  objetivo.
\item
  \textbf{Formulación de hipótesis.} Tal como se realiza en el método
  científico, formular hipótesis desde varias perspectivas ante los
  datos, puede contribuir a la obtención de todas las alternativas
  viables que favorezcan el análisis.
\item
  \textbf{Análisis de datos.} Utilizar las herramientas y técnicas
  vistas anteriormente. Si no se está familiarizado con ellas, se puede
  comenzar con software comerciales, mientras que al mismo tiempo se
  realizan pruebas con las tecnologías nuevas para observar las
  diferencias y usabilidades. Utilizar herramientas de análisis de
  datos, para examinar y procesar los datos recopilados. Realizar
  análisis estadísticos, crear gráficos y tablas para visualizar los
  datos y encontrar patrones, tendencias o ``insights'' relevantes.
\item
  \textbf{Interpretación de resultados.} Es importante cerciorarse en
  todo momento de no haber cometido una omisión o un error de análisis.
  En este punto es recomendable un externo crítico que pueda orientar y
  dar su punto de vista.
\item
  \textbf{Comunicación y documentación de resultados.}
\item
  \textbf{Tomar decisiones informadas.} Basándose en los datos y
  análisis realizados, tomar decisiones informadas sobre la viabilidad
  de la idea a emprender teniendo presente que es un proceso iterativo,
  y es posible que se necesite revisar y ajustar las suposiciones y
  análisis en la medida en la que se avanza en el proyecto.
\end{itemize}
\end{quote}

Al display en donde se exhiben y concentran los indicadores,
cotidianamente se le conoce como Dashboard o Cuadro de Mando Integral.
El emprendedor podrá ir determinando qué tantos indicadores necesitará
para conducir cómodamente su proyecto o empresa.

\bookmarksetup{startatroot}

\hypertarget{proceso-emprendedor}{%
\chapter{Proceso Emprendedor}\label{proceso-emprendedor}}

Autora: Karen Jualiana Fernández Castillo

Lorem ipsum dolor sit amet, consectetur adipiscing elit. Praesent
dapibus ut libero nec semper. In quam lorem, rutrum in nulla quis,
elementum volutpat odio. Phasellus felis nunc, semper eu sodales eu,
laoreet nec arcu. Sed ac magna quis sapien accumsan gravida. Etiam
tristique dui id elit egestas condimentum. Integer gravida fermentum
placerat. Ut hendrerit viverra ipsum, id vehicula magna tincidunt sed.
Nunc fermentum diam purus, non dignissim purus tincidunt vitae. Sed
tincidunt tortor vitae malesuada molestie. Sed aliquet, est a vulputate
aliquet, massa erat hendrerit arcu, in ultrices nulla nisl vel tortor.
Nam velit est, venenatis et tortor ac, rhoncus feugiat est. Nunc non
neque erat. Etiam eget ipsum fermentum risus lobortis consequat sit amet
sit amet dui. Maecenas auctor vehicula volutpat.

\hypertarget{titular-2}{%
\section{Titular}\label{titular-2}}

Etiam ultricies magna imperdiet nunc malesuada, ac lobortis sapien
rhoncus. Aenean eros lectus, accumsan vitae faucibus a, aliquam ac diam.
Maecenas finibus justo non nibh pharetra aliquam. Maecenas egestas, ex
vitae blandit convallis, leo mauris ornare nunc, porta fermentum tellus
est et urna. Aenean eu est et sapien laoreet placerat. Nunc turpis
ipsum, dapibus a ipsum at, tempus pretium lacus. In libero turpis,
tristique id orci a, auctor rhoncus risus. Praesent lacus nunc,
sollicitudin quis ipsum id, pulvinar venenatis mi.

\hypertarget{titular-3}{%
\subsection{Titular}\label{titular-3}}

Vivamus libero leo, accumsan non turpis vel, eleifend sodales mauris.
Donec pellentesque, tellus a lacinia vestibulum, nisl lacus dapibus
nibh, id tincidunt turpis erat non nisl. Donec ornare imperdiet metus,
eu sollicitudin risus elementum non. Aliquam metus eros, maximus
dignissim pellentesque nec, venenatis vitae purus. Vivamus laoreet mi
magna, ac malesuada tellus condimentum in. Integer nulla lectus, finibus
sit amet ex nec, finibus molestie diam. Pellentesque gravida ac erat sit
amet tempus.

\bookmarksetup{startatroot}

\hypertarget{pruxe1cticas-de-gestiuxf3n}{%
\chapter{Prácticas de Gestión}\label{pruxe1cticas-de-gestiuxf3n}}

Autora: Karen Paulina Salazar Núñez

Lorem ipsum dolor sit amet, consectetur adipiscing elit. Praesent
dapibus ut libero nec semper. In quam lorem, rutrum in nulla quis,
elementum volutpat odio. Phasellus felis nunc, semper eu sodales eu,
laoreet nec arcu. Sed ac magna quis sapien accumsan gravida. Etiam
tristique dui id elit egestas condimentum. Integer gravida fermentum
placerat. Ut hendrerit viverra ipsum, id vehicula magna tincidunt sed.
Nunc fermentum diam purus, non dignissim purus tincidunt vitae. Sed
tincidunt tortor vitae malesuada molestie. Sed aliquet, est a vulputate
aliquet, massa erat hendrerit arcu, in ultrices nulla nisl vel tortor.
Nam velit est, venenatis et tortor ac, rhoncus feugiat est. Nunc non
neque erat. Etiam eget ipsum fermentum risus lobortis consequat sit amet
sit amet dui. Maecenas auctor vehicula volutpat.

\hypertarget{titular-4}{%
\section{Titular}\label{titular-4}}

Etiam ultricies magna imperdiet nunc malesuada, ac lobortis sapien
rhoncus. Aenean eros lectus, accumsan vitae faucibus a, aliquam ac diam.
Maecenas finibus justo non nibh pharetra aliquam. Maecenas egestas, ex
vitae blandit convallis, leo mauris ornare nunc, porta fermentum tellus
est et urna. Aenean eu est et sapien laoreet placerat. Nunc turpis
ipsum, dapibus a ipsum at, tempus pretium lacus. In libero turpis,
tristique id orci a, auctor rhoncus risus. Praesent lacus nunc,
sollicitudin quis ipsum id, pulvinar venenatis mi.

\hypertarget{titular-5}{%
\subsection{Titular}\label{titular-5}}

Vivamus libero leo, accumsan non turpis vel, eleifend sodales mauris.
Donec pellentesque, tellus a lacinia vestibulum, nisl lacus dapibus
nibh, id tincidunt turpis erat non nisl. Donec ornare imperdiet metus,
eu sollicitudin risus elementum non. Aliquam metus eros, maximus
dignissim pellentesque nec, venenatis vitae purus. Vivamus laoreet mi
magna, ac malesuada tellus condimentum in. Integer nulla lectus, finibus
sit amet ex nec, finibus molestie diam. Pellentesque gravida ac erat sit
amet tempus.

\bookmarksetup{startatroot}

\hypertarget{oportunidades}{%
\chapter{Oportunidades}\label{oportunidades}}

Autora: Jenny María Checo Vargas

Temas de referencia: - a - b - c

Lorem ipsum dolor sit amet, consectetur adipiscing elit. Praesent
dapibus ut libero nec semper. In quam lorem, rutrum in nulla quis,
elementum volutpat odio. Phasellus felis nunc, semper eu sodales eu,
laoreet nec arcu. Sed ac magna quis sapien accumsan gravida. Etiam
tristique dui id elit egestas condimentum. Integer gravida fermentum
placerat. Ut hendrerit viverra ipsum, id vehicula magna tincidunt sed.
Nunc fermentum diam purus, non dignissim purus tincidunt vitae. Sed
tincidunt tortor vitae malesuada molestie. Sed aliquet, est a vulputate
aliquet, massa erat hendrerit arcu, in ultrices nulla nisl vel tortor.
Nam velit est, venenatis et tortor ac, rhoncus feugiat est. Nunc non
neque erat. Etiam eget ipsum fermentum risus lobortis consequat sit amet
sit amet dui. Maecenas auctor vehicula volutpat.

\hypertarget{titular-6}{%
\section{Titular}\label{titular-6}}

Etiam ultricies magna imperdiet nunc malesuada, ac lobortis sapien
rhoncus. Aenean eros lectus, accumsan vitae faucibus a, aliquam ac diam.
Maecenas finibus justo non nibh pharetra aliquam. Maecenas egestas, ex
vitae blandit convallis, leo mauris ornare nunc, porta fermentum tellus
est et urna. Aenean eu est et sapien laoreet placerat. Nunc turpis
ipsum, dapibus a ipsum at, tempus pretium lacus. In libero turpis,
tristique id orci a, auctor rhoncus risus. Praesent lacus nunc,
sollicitudin quis ipsum id, pulvinar venenatis mi.

\hypertarget{titular-7}{%
\subsection{Titular}\label{titular-7}}

Vivamus libero leo, accumsan non turpis vel, eleifend sodales mauris.
Donec pellentesque, tellus a lacinia vestibulum, nisl lacus dapibus
nibh, id tincidunt turpis erat non nisl. Donec ornare imperdiet metus,
eu sollicitudin risus elementum non. Aliquam metus eros, maximus
dignissim pellentesque nec, venenatis vitae purus. Vivamus laoreet mi
magna, ac malesuada tellus condimentum in. Integer nulla lectus, finibus
sit amet ex nec, finibus molestie diam. Pellentesque gravida ac erat sit
amet tempus.

\bookmarksetup{startatroot}

\hypertarget{uxe9tica-de-datos-y-emprendimiento}{%
\chapter{Ética de Datos y
Emprendimiento}\label{uxe9tica-de-datos-y-emprendimiento}}

Autora: Mensia Indira Otáñez de la Rosa

Lorem ipsum dolor sit amet, consectetur adipiscing elit. Praesent
dapibus ut libero nec semper. In quam lorem, rutrum in nulla quis,
elementum volutpat odio. Phasellus felis nunc, semper eu sodales eu,
laoreet nec arcu. Sed ac magna quis sapien accumsan gravida. Etiam
tristique dui id elit egestas condimentum. Integer gravida fermentum
placerat. Ut hendrerit viverra ipsum, id vehicula magna tincidunt sed.
Nunc fermentum diam purus, non dignissim purus tincidunt vitae. Sed
tincidunt tortor vitae malesuada molestie. Sed aliquet, est a vulputate
aliquet, massa erat hendrerit arcu, in ultrices nulla nisl vel tortor.
Nam velit est, venenatis et tortor ac, rhoncus feugiat est. Nunc non
neque erat. Etiam eget ipsum fermentum risus lobortis consequat sit amet
sit amet dui. Maecenas auctor vehicula volutpat.

\hypertarget{titular-8}{%
\section{Titular}\label{titular-8}}

Etiam ultricies magna imperdiet nunc malesuada, ac lobortis sapien
rhoncus. Aenean eros lectus, accumsan vitae faucibus a, aliquam ac diam.
Maecenas finibus justo non nibh pharetra aliquam. Maecenas egestas, ex
vitae blandit convallis, leo mauris ornare nunc, porta fermentum tellus
est et urna. Aenean eu est et sapien laoreet placerat. Nunc turpis
ipsum, dapibus a ipsum at, tempus pretium lacus. In libero turpis,
tristique id orci a, auctor rhoncus risus. Praesent lacus nunc,
sollicitudin quis ipsum id, pulvinar venenatis mi.

\hypertarget{titular-9}{%
\subsection{Titular}\label{titular-9}}

Vivamus libero leo, accumsan non turpis vel, eleifend sodales mauris.
Donec pellentesque, tellus a lacinia vestibulum, nisl lacus dapibus
nibh, id tincidunt turpis erat non nisl. Donec ornare imperdiet metus,
eu sollicitudin risus elementum non. Aliquam metus eros, maximus
dignissim pellentesque nec, venenatis vitae purus. Vivamus laoreet mi
magna, ac malesuada tellus condimentum in. Integer nulla lectus, finibus
sit amet ex nec, finibus molestie diam. Pellentesque gravida ac erat sit
amet tempus.

\bookmarksetup{startatroot}

\hypertarget{finanzas-de-datos-para-emprendedores}{%
\chapter{Finanzas de Datos para
Emprendedores}\label{finanzas-de-datos-para-emprendedores}}

Autora: Salomé Shirley Criollo Linaico

Lorem ipsum dolor sit amet, consectetur adipiscing elit. Praesent
dapibus ut libero nec semper. In quam lorem, rutrum in nulla quis,
elementum volutpat odio. Phasellus felis nunc, semper eu sodales eu,
laoreet nec arcu. Sed ac magna quis sapien accumsan gravida. Etiam
tristique dui id elit egestas condimentum. Integer gravida fermentum
placerat. Ut hendrerit viverra ipsum, id vehicula magna tincidunt sed.
Nunc fermentum diam purus, non dignissim purus tincidunt vitae. Sed
tincidunt tortor vitae malesuada molestie. Sed aliquet, est a vulputate
aliquet, massa erat hendrerit arcu, in ultrices nulla nisl vel tortor.
Nam velit est, venenatis et tortor ac, rhoncus feugiat est. Nunc non
neque erat. Etiam eget ipsum fermentum risus lobortis consequat sit amet
sit amet dui. Maecenas auctor vehicula volutpat.

\hypertarget{titular-10}{%
\section{Titular}\label{titular-10}}

Etiam ultricies magna imperdiet nunc malesuada, ac lobortis sapien
rhoncus. Aenean eros lectus, accumsan vitae faucibus a, aliquam ac diam.
Maecenas finibus justo non nibh pharetra aliquam. Maecenas egestas, ex
vitae blandit convallis, leo mauris ornare nunc, porta fermentum tellus
est et urna. Aenean eu est et sapien laoreet placerat. Nunc turpis
ipsum, dapibus a ipsum at, tempus pretium lacus. In libero turpis,
tristique id orci a, auctor rhoncus risus. Praesent lacus nunc,
sollicitudin quis ipsum id, pulvinar venenatis mi.

\hypertarget{titular-11}{%
\subsection{Titular}\label{titular-11}}

Vivamus libero leo, accumsan non turpis vel, eleifend sodales mauris.
Donec pellentesque, tellus a lacinia vestibulum, nisl lacus dapibus
nibh, id tincidunt turpis erat non nisl. Donec ornare imperdiet metus,
eu sollicitudin risus elementum non. Aliquam metus eros, maximus
dignissim pellentesque nec, venenatis vitae purus. Vivamus laoreet mi
magna, ac malesuada tellus condimentum in. Integer nulla lectus, finibus
sit amet ex nec, finibus molestie diam. Pellentesque gravida ac erat sit
amet tempus.

\bookmarksetup{startatroot}

\hypertarget{conocimiento-del-cliente-a-travuxe9s-de-los-datos}{%
\chapter{Conocimiento del Cliente a través de los
Datos}\label{conocimiento-del-cliente-a-travuxe9s-de-los-datos}}

Autora: Claudia Reina Vargas

\hypertarget{introducciuxf3n-1}{%
\section{Introducción}\label{introducciuxf3n-1}}

El sector turístico, se trata de un sector que, debido a la
globalización y la revolución digital del sistema entre otros factores a
destacar, se ha vuelto altamente competitivo. Por ello ha surgido la
necesidad de formular nuevas estrategias de marketing y análisis de
datos para la reinvención del sector en un destino en concreto. La alta
competencia y las facilidades que tiene el turista a la hora de poder
seleccionar el destino son claves para el crecimiento de este sector. Si
hablamos del caso de España, según los datos del Ministerio de Industria
Comercio y Turismo, en 2022 se superaron las cifras prepandemia por la
Covid-19, alcanzando los 71,6 millones de turistas internacionales que
realizaron un gasto de 87.061 millones de euros, provenientes de Reino
Unido, Alemania y Francia principalmente. En cuanto al gasto, los datos
arrojan un aumento del 10,5\% de media por turista frente a 2019 y las
estancias aumentaron a una media de 7,5 días de media. Observando estas
cifras, podemos ver que el turismo está sufriendo un gran auge, pero
¿hasta cuándo durará esto? El aumento del coste de vida ha sido notable
en muchas industrias, y puede, que ciertos modelos de turismo en países
estacionalizados no sea viable en un futuro. Ante este escenario
cambiante y de gran incertidumbre, es necesario que el sector del
turismo desarrolle continuamente nuevos proyectos para seguir siendo una
industria sólida, con una consolidada posición de liderazgo
internacional y con un potencial incansable (Consejo Español de Turismo,
2007). Muchas empresas, en sus inicios, estaban presentes únicamente en
un mercado concreto y en una tipología concreta de turismo, pero se
vieron obligadas a lanzarse al mercado internacional para poder
subsistir al crecimiento de otras entidades extranjeras. Pero el propio
consumidor ha cambiado también sus hábitos de compra y consumo de
servicios, teniendo mayor poder de decisión y negociación que las
propias empresas. A diario se reciben miles de estímulos que impulsan a
tomar unas u otras decisiones de compra. La investigación de mercados
constituye una parte fundamental para desarrollo de estrategias de
crecimiento y mejora, pero junto con un análisis correcto se deben
emplear otras herramientas y técnicas adecuadas. En este sentido, el uso
de big data, la minería de datos o técnicas de Business Intelligence
(BI) suponen una importante oportunidad para las empresas turísticas que
quieran aumentar sus ventas tanto para el público nacional como para el
internacional, ayudando con la innovación y el posicionamiento de los
destinos en diferentes mercados, conocimiento de la demanda y del
público objetivo entre otros factores.

\hypertarget{fuentes-de-datos-utilizadas-en-el-sector-turuxedstico}{%
\section{Fuentes de datos utilizadas en el sector
turístico}\label{fuentes-de-datos-utilizadas-en-el-sector-turuxedstico}}

El rápido crecimiento de las Tecnologías de la Información y la
Comunicación (TICs) ha generado grandes transformaciones sociales y
económicas a nivel mundial. En el sector turístico, se utilizan diversas
técnicas de análisis de datos para obtener información valiosa que
permita tomar decisiones estratégicas y mejorar la experiencia de los
clientes, así como para lograr el posicionamiento de un destino. Algunas
de estas técnicas incluyen la minería de datos, el aprendizaje
automático, la inteligencia artificial y la segmentación de clientes,
para cuando se tiene acceso a estas herramientas, pero tambiíen hay
otras como los datos proporcionados por webs oficiales de Goviernos y
Ministerios, las redes sociales (como Facebook, Twitter, Instagram o
Tripadvisor), diferentes estudios de mercado públicos en el sector que
deseen,etc., las cuales iremos desglosando para identificar sus ventajes
y aportes a esta industria. Estos sitemas ofrecen la información
necesaria a las empresas para tomar decisiones que permitan aumentar la
capacidad productiva, impulsar la competitividad y evaluar el retorno de
las acciones de fomento de la actividad turística. Por otra parte,
contribuye a incrementar las prestaciones a los visitantes adaptando la
oferta a las nuevas exigencias y necesidades. También sirve de ayuda a
los gestores de proyectos de destinos turísticos o para la correcta
elaboración de estrategias y planes de marketing o planes de acción.
Otro de sus usos es para averiguar el por qué un destino atrae más a un
determinado segmento y no por otro o los precios que pueden poner según
los departamentos de revenue. En este capítulo, se procederá a realizar
un desglose con mayor profundidad de cuáles son las herramientas que
ayudan en el sector, así como las aplicaciones que tienen para la
obtención de datos, sus diferencias y un caso práctico de una empresa
turística que recurra a estas herramientas para un análisis correcto. El
objetivo es poner en valor aquellas herramientas que con el paso del
tiempo se han podido desarrollar gracias a las TICs y como su uso mejora
y facilita las decisiones de las empresas, no solo del sector turístico,
sino de todos los implicados en la cadena de valor de un servicio, para
posteriormente centrarnos en como conseguir datos por fuera de estas
herramientas.

\hypertarget{smart-data}{%
\subsection{Smart Data}\label{smart-data}}

Para los emprendedores, la manera más sencilla de de otender datos sobre
el sector para emprendedores en la industria es mediante informes y
webs/apps y Smart Data. Este concepto se refiere a la utilización de
datos relevantes y significativos para obtener información y
conocimientos valiosos. Para los emprendedores en el sector turístico,
el smart data puede ser una herramienta poderosa para comprender mejor a
los clientes, tomar decisiones informadas y mejorar la eficiencia
operativa. Entre estas formas de búsqueda y aplicación podemos destacar,
por ejemplo, análisis de datos de clientes mediante informes
demográficos proporcionados de manera gratuita por diferentes entidades.
En el caso de España podemos destacar, el Instituto Nacional de
Estadística (INE), DATAESTUR (Selección de los datos más relevantes del
sector turístico), SEGITTUR (sociedad estatal española dedicada a la
gestión de la innovación y las tecnologías turísticas), EXCELTUR, y
muchas más. En sus webs se pueden encontrar análisis e informes del
sector, los cuales son de uso público y están a disposición de todo el
mundo, permitiendo así realizar análisis de referencias de viaje,
historial de reservas, demografía, comentarios y opiniones, para
comprender mejor a su audiencia. Por otro lado, se puede realizar un
monitoreo de redes sociales. Las redes sociales son una fuente valiosa
de smart data para los emprendedores turísticos. Al monitorear las
conversaciones en las redes sociales, las ubicaciones más usadas o en
tendencia, las fotografías de otros usuarios o de su competencia o
realizar encuestas los emprendedores pueden obtener información en
tiempo real sobre las opiniones de los clientes, las tendencias de
viaje, los destinos populares y las experiencias compartidas por los
viajeros. Esto puede ayudar a ajustar sus estrategias de marketing,
identificar oportunidades y responder rápidamente a las necesidades y
expectativas de los clientes. Por último, mediante el análisis de
opiniones web. Este hecho cada vez tiene más importancia tanto para
usuarios como para los empresarios, que buscan en ambos casos mostrar la
realidad que hay detrás de un negocio mediante opiniones públicas, a las
que tiene acceso todo el mundo, y las cuales se quedan guardadas en los
sitios webs, pudiendo ir acompañadas en algunos casos de vídeos o
fotografías. Al obtener esta información de su negocio y de sus
competidores, los emprendedores pueden identificar áreas de mejora,
tomar medidas correctivas y mantener la satisfacción del cliente en un
nivel alto. Pero estas no son las únicas tecnologías o pasos que puede
seguir un emprendedor para analizar al turista o el sector, existen como
ya hemos mencionado tecnologías más potentes, las cuales ahorran tiempo
y generan informes de toda la información, pudiendo conseguir una
adaptación de estrategias, segmentaciones y ajuste de precios entre
otros factores más rápida.

\hypertarget{mineruxeda-de-datos}{%
\subsection{Minería de Datos}\label{mineruxeda-de-datos}}

En los últimos años, debido al desarrollo exponencial de la tecnología,
se genera cada vez más información, la cual es más difícil de analizar,
pero a su vez el acceso a ella es más sencillo. Estas grandes cantidades
de datos disponibles online o almacenados en bases de datos, son una
oportunidad a la hora de obtener nueva información que es potencialmente
importante pero que aún no ha sido descubierta. Se hace referencia al
concepto de ``Big Data'' cuando esta información, debido a su volumen y
complejidad, no puede ser procesada o analizada utilizando procesos o
herramientas tradicionales. Berman (2013) propone que se defina Big Data
a partir de las tres V's: volumen, variedad y velocidad. Cuando habla de
volumen se refiere a contar con una gran cantidad de datos. La variedad
está dada por las diferentes formas y formatos que tienen esos datos, ya
sea en bases de datos tradicionales, imágenes, documentos y registros
complejos. Por último, la velocidad hace alusión a que el contenido de
los datos cambia constantemente, ya sea a través de la generación de
nuevos datos, de la absorción de colecciones de datos complementarias o
de la introducción de datos previamente archivados. El análisis de esta
información suele realizarse por personas expertas en la materia, pero
este proceso tiene altos costes monetarios y requiere de mucho tiempo.
Es aquí cuando entra el juego la minería de datos o data mining, ya que
esta tecnología se sirve de estos sistemas utilizando algoritmos
específicos para extraer patrones desde los datos, permitiendo encontrar
patrones repetitivos que se encuentran escondidos en esos datos. Para
ello combina la estadística, la inteligencia artificial, el aprendizaje
automático (machine learning) y la tecnología de bases de datos.
\includegraphics{index_files/mediabag/ebf9853f-4644-4ba6-a.pdf} Una vez
extraída la información de la minería de datos, se puede usar para
predicciones de comportamientos, demandas y ventajas competitivas.
\includegraphics{index_files/mediabag/539771e1-4a7c-4164-8.pdf} En
cuanto a su aplicación en el sector turístico podemos destacar
diferentes posibilidades: Por un lado se utiliza para el análisis de
datos de reservas de hoteles para identificar los patrones de reserva,
como la antelación con la que los clientes realizan sus reservas, las
temporadas de mayor demanda, los tipos de habitación más populares, etc.
Al analizar los datos de reservas de hoteles, se puede determinar la
antelación con la que los clientes realizan sus reservas. Esto permite
identificar tendencias y patrones en cuanto al período de tiempo entre
la fecha de reserva y la fecha de llegada. Por ejemplo, se puede
descubrir que los viajeros de negocios tienden a reservar con menos
antelación que los turistas de ocio. También permite identificar las
temporadas de mayor demanda en la industria hotelera. Al analizar las
fluctuaciones en las reservas a lo largo del tiempo, se pueden
identificar patrones estacionales y determinar cuándo hay una mayor
demanda de habitaciones. Esta información es esencial para establecer
estrategias de precios y planificación de la capacidad. Otro beneficio
sería la duración de la estancia o el canal de la reserva. En cuanto a
la duración de la estancia, el data mining sirve como herramienta de
elaboración de informes de conducta, diferenciación de tipologías de
clientes y predicciones a la hora de establecer precios. Además, ayuda a
recordar eventos que han sucedido en lugares concretos y por los cuales
las estancias se vieron afectadas, pudiendo alargarse o acortarse, y
sirve como base de datos para años futuros. Por otro lado, analizar los
canales de reserva puede ayudar en el proceso de establecimiento de una
estrategia, en la cual e puede ver que canal es el más usado, ya sea
online mediante OTAs como Booking, Expedia, Hotelbeds, etc., recurriendo
a los servicios de reserva directos por parte del alojamiento/servicio
como su propia web o mediante un CRS concreto (programa de central de
reservas u oficinas donde gestionan las mismas, ya sea mediante email o
teléfono).

\hypertarget{inteligencia-artificial-ia}{%
\subsection{Inteligencia Artificial
(IA)}\label{inteligencia-artificial-ia}}

La inteligencia artificial (IA) es un campo de estudio y desarrollo que
se enfoca en crear sistemas capaces de realizar tareas que normalmente
requerirían de la inteligencia humana. La idea principal detrás de la IA
es emular ciertos aspectos de la capacidad humana para aprender,
razonar, comprender, planificar, tomar decisiones y resolver problemas.
Se basa en la idea de que las máquinas pueden procesar información de
manera similar a los seres humanos, y utilizarla para tomar decisiones o
realizar acciones de manera autónoma. Los sistemas de IA pueden aprender
de datos, adaptarse a nuevas situaciones y mejorar su rendimiento a lo
largo del tiempo, sin necesidad de ser programados de forma explícita
para cada tarea específica. Existen diferentes enfoques y técnicas
dentro del campo de la IA. Uno de los enfoques más utilizados es el
aprendizaje automático (machine learning), que se basa en algoritmos y
modelos matemáticos que permiten a las máquinas aprender a través de la
experiencia y los datos, del cual hablaremos más adelante. También
existen otros enfoques dentro de la IA, los cuales incluyen el
procesamiento de lenguaje natural (NLP, por sus siglas en inglés), que
se enfoca en la comprensión y generación de lenguaje humano, la visión
por ordenador, que permite a las máquinas analizar y comprender imágenes
y videos, la robótica, que busca desarrollar robots capaces de
interactuar con el entorno de manera inteligente y la planificación y
toma de decisiones, que involucra la capacidad de los sistemas de IA
para tomar decisiones óptimas en situaciones complejas. La IA se utiliza
en una amplia variedad de industrias y campos, incluyendo el sector
turístico. En el turismo, la IA ha encontrado aplicaciones en áreas como
la atención al cliente, la personalización de ofertas, la gestión de
destinos, el análisis de datos de viajeros, la mejora de la seguridad y
la optimización de precios. A través de chatbots y asistentes virtuales,
las empresas turísticas pueden brindar atención al cliente las 24 horas
del día, los 7 días de la semana, respondiendo preguntas y ofreciendo
recomendaciones personalizadas. Mediante el análisis de datos de los
clientes, la IA permite a las empresas comprender mejor las preferencias
y necesidades de los viajeros, ofreciendo ofertas y promociones más
relevantes. En la gestión de destinos, la IA puede analizar datos de
geolocalización y comportamiento de los turistas para mejorar la
planificación de rutas, identificar áreas de interés y optimizar la
experiencia del viajero. Además, la IA también se utiliza en la
detección de fraudes y la mejora de la seguridad en el sector turístico,
ayudando a identificar actividades sospechosas y proteger tanto a los
turistas como a las empresas. Se utiliza también para analizar
comentarios y reseñas de clientes en plataformas en línea. Al aplicar
técnicas de procesamiento de lenguaje natural, la IA puede identificar
el tono emocional de los comentarios (positivo, negativo o neutral) y
extraer información útil sobre las experiencias de los clientes. Esto
ayuda a los proveedores de servicios turísticos a comprender mejor las
necesidades y expectativas de sus clientes. Por otro lado, debido al
auge de los dispositivos móviles y las aplicaciones de viaje, que
generan grandes cantidades de datos de geolocalización, gracias a la IA
se puede analizar estos datos para comprender los patrones de movilidad
de los turistas, identificar lugares de interés populares y pronosticar
la congestión de tráfico en ciertas áreas. Esto ayuda a los proveedores
de servicios turísticos a mejorar la planificación de rutas y optimizar
la gestión de destinos. Por último, en cuanto al diseño de estrategias
de marketing, la IA puede identificar patrones y segmentar a los
clientes en grupos con características similares, esto se hace mediante
el procesamiento de grandes cantidades de datos de clientes, como
preferencias de viaje, comportamiento de compra, historial de reservas,
interacciones en redes sociales, entre otros, lo que ayuda a la creación
de estas campañas por segmentos para que tengan mayor efectividad. El
objetivo principal de la segmentación de clientes es comprender mejor a
los diferentes grupos de clientes y adaptar las estrategias de marketing
y servicios para satisfacer sus necesidades de manera más efectiva. La
segmentación de clientes se basa en la recopilación y análisis de datos
relevantes, como datos demográficos, preferencias de viaje,
comportamiento de compra, historial de reservas, entre otros. Estos
datos se utilizan para identificar patrones y tendencias en el
comportamiento de los clientes, lo que permite agruparlos en segmentos
con características similares.

La inteligencia artificial (IA) desempeña un papel importante en la
segmentación de clientes, ya que puede procesar grandes volúmenes de
datos y encontrar patrones y relaciones complejas de manera más
eficiente que los métodos tradicionales. Los algoritmos de IA utilizados
en la segmentación de clientes pueden aplicar técnicas de aprendizaje
automático para identificar relaciones no lineales y comprender las
preferencias y comportamientos de los clientes de manera más precisa.
Tal y como habíamos mencionado antes, La inteligencia artificial y el
aprendizaje automático (machine learning) están estrechamente
relacionados y se considera que el aprendizaje automático es una subrama
de la IA. La inteligencia artificial, por su lado, se refiere al campo
de estudio y desarrollo de sistemas que pueden realizar tareas que
normalmente requerirían de la inteligencia humana buscando emular
ciertos aspectos de la inteligencia humana, mientras que el aprendizaje
automático, por otro lado, es una técnica dentro de la IA que se basa en
algoritmos y modelos matemáticos para permitir que las máquinas aprendan
de datos y experiencias. El objetivo del aprendizaje automático es
desarrollar modelos y algoritmos que puedan aprender a reconocer
patrones, hacer predicciones, tomar decisiones y resolver problemas a
través de la experiencia y los datos En otras palabras, el aprendizaje
automático es una herramienta utilizada en la implementación de la
inteligencia artificial.

\hypertarget{aprendizaje-automuxe1tico}{%
\subsection{Aprendizaje automático}\label{aprendizaje-automuxe1tico}}

El aprendizaje automático, también conocido como machine learning, es
una técnica ampliamente utilizada en el sector turístico para el
análisis de datos y la toma de decisiones informadas, siendo una
subdivisión de la Inteligencia Artificial que se basa en algoritmos para
analizar conjuntos de datos masivos. Los datos se convierten en
información, conocimiento y por último en una herramienta de sabiduría
para realizar acciones con una base sólida. Estos algoritmos permiten a
las máquinas aprender y mejorar automáticamente a partir de los datos
sin ser programadas explícitamente. El objetivo del Machine Learning es
desarrollar sistemas que puedan analizar y comprender grandes cantidades
de datos, identificar patrones, hacer predicciones o tomar decisiones
basadas en esos patrones identificados. En lugar de seguir reglas y
algoritmos predefinidos, el Machine Learning se basa en la capacidad de
las máquinas para aprender de los datos y ajustar sus acciones o
predicciones en función de esa experiencia. El proceso de Machine
Learning generalmente se divide en tres etapas principales: -
Entrenamiento: Durante esta etapa, se proporciona a los algoritmos de
Machine Learning un conjunto de datos de entrenamiento que contiene
ejemplos y respuestas esperadas. El algoritmo utiliza estos datos para
aprender patrones y crear un modelo que pueda hacer predicciones o tomar
decisiones. - Prueba y validación: Una vez que se ha entrenado el
modelo, se prueba su rendimiento utilizando un conjunto de datos de
prueba independiente. Este paso es importante para verificar si el
modelo puede hacer predicciones precisas o tomar decisiones acertadas. -
Implementación y mejora: Después de que el modelo ha sido probado y
validado, se implementa en un entorno de producción para su uso real. A
medida que se recopilan nuevos datos, el modelo puede seguir mejorando y
ajustándose mediante técnicas como el entrenamiento continuo o el
aprendizaje en línea. Existen diferentes enfoques y técnicas que se
utilizan para desarrollar modelos y algoritmos: - Aprendizaje
supervisado: En este enfoque, el modelo se entrena utilizando un
conjunto de datos etiquetados, es decir, datos que contienen ejemplos y
sus respuestas correspondientes. El objetivo es que el modelo aprenda a
hacer predicciones o clasificar nuevos datos basándose en los patrones
identificados en el conjunto de entrenamiento. - Aprendizaje no
supervisado: En este caso, el modelo se entrena utilizando un conjunto
de datos no etiquetados, es decir, datos sin respuestas previas. El
objetivo es que el modelo descubra patrones y estructuras ocultas en los
datos, agrupándolos o realizando otras tareas de análisis sin necesidad
de respuestas predefinidas. - Aprendizaje por refuerzo: En este enfoque,
el modelo aprende a través de la interacción con un entorno y la
retroalimentación recibida. El modelo toma acciones en un entorno y
recibe recompensas o penalizaciones según su desempeño. El objetivo es
que el modelo aprenda a tomar decisiones óptimas para maximizar la
recompensa a largo plazo. - Aprendizaje profundo (Deep Learning): Es una
técnica que se basa en redes neuronales artificiales profundas para
aprender y extraer características complejas de los datos. Las redes
neuronales profundas están compuestas por múltiples capas y pueden
aprender representaciones de alto nivel de los datos, lo que las hace
especialmente útiles para tareas como el reconocimiento de imágenes, el
procesamiento del lenguaje natural y la traducción automática Si
hablamos de la aplicación de esta metodología para el sector turístico
podemos destacar diferentes usos. Una de las más conocidas sería la
personalización de recomendaciones ya que los algoritmos de aprendizaje
automático se utilizan para analizar los datos de los usuarios, como sus
preferencias, historial de reservas y comportamiento de navegación, con
el fin de ofrecer recomendaciones personalizadas. Estos algoritmos
pueden identificar patrones y tendencias para ofrecer opciones de viaje
relevantes y adaptadas a las preferencias individuales de los usuarios,
algo que el usuario cada vez nota más, ya que cuando busca un
alojamiento, puede que le aparezcan primero los resultados de otras
cadenas en las que se ha alojado o compañías aéreas con las que ha
volado. La demanda futura también se puede predecir con esta técnica, ya
que analiza grandes volúmenes de datos históricos, así como datos en
tiempo real, para predecir la demanda futura y ajustar los precios de
manera óptima. Esto ayuda a las empresas turísticas a establecer precios
competitivos, maximizar los ingresos y gestionar la capacidad de manera
eficiente. Uno de los usos más recientes de esta metodología es la
aplicación de algoritmos para analizar comentarios de los usuarios en
redes sociales, sitios web de reseñas y otras fuentes de
retroalimentación. Esto permite a las empresas turísticas comprender las
opiniones y los sentimientos de los clientes, identificar tendencias y
áreas de mejora, y responder de manera oportuna a los comentarios, pero
también ayuda con la detección de fraudes en reservas, identificando
patrones y comportamientos sospechosos en las transacciones. Esto ayuda
a prevenir actividades fraudulentas y proteger tanto a las empresas
turísticas como a los clientes. Además, se utiliza en sistemas de
seguridad para detectar amenazas y riesgos potenciales. En general, el
aprendizaje automático ofrece beneficios significativos en el sector
turístico, permitiendo a las empresas tomar decisiones más eficientes y
personalizadas, optimizar la experiencia del cliente, gestionar mejor
los recursos y adaptarse rápidamente a las tendencias y cambios del
mercado.

\hypertarget{big-data}{%
\subsection{Big Data}\label{big-data}}

De acuerdo con M. Chen et al (2014) y según se cita en Lukosius y Hyman
(2018), el big data son ``conjuntos de datos que no podrían percibirse,
adquirirse, gestionarse y procesarse por medio de TICs tradicionales y
herramientas de software/hardware dentro de unos tiempos razonables''.
Para el mundo empresarial, el Big Data ha cobrado más y más importancia,
el cual les da información vital sobre los consumidores, como sus
gustos, sentimientos, opiniones, preferencias, comportamientos
recurrentes, etc., extrayendo conclusiones de cantidades ingentes de
información para su posterior análisis. Estos conjuntos de datos son
generados a gran velocidad y en variedad de formatos, como datos
estructurados (por ejemplo, bases de datos), datos no estructurados
(como texto, imágenes o videos) y datos semiestructurados (como archivos
XML o JSON). En el sector turístico, el Big Data juega un papel
fundamental en la obtención y análisis de datos de los clientes. A
medida que los avances tecnológicos y la digitalización han transformado
la forma en que los turistas interactúan con las empresas y los destinos
turísticos, se ha generado una gran cantidad de datos relacionados con
los viajes y las experiencias de los clientes. En el caso del Big Data,
presenta unas características más particulares que las del resto de
herramientas. Por un lado, podemos destacar el volumen, y es que los
conjuntos de datos pueden abarcar desde terabytes hasta petabytes o
incluso exabytes de información. Esta cantidad de datos proviene de
diversas fuentes, como redes sociales, sensores, transacciones,
registros, entre otros. Otro aspecto es la velocidad, dado que los datos
son generados a y actualizados a altas velocidades, y esque en este tipo
de herramienta se requiere una respuesta rápida para aprovechar las
oportunidades o solucionar problemas. Como ya hemos mencionado, el Big
Data abarca diferentes tipos de datos y esto hace que se planteen
desafíos adicionales para su almacenamiento y procesamiento. Por último,
se debe tener en cuenta que los datos pueden ser incompletos,
inconsistentes o incorrectos. Debido a esto, la veracidad de los datos
puede ser un desafío, ya que es necesario validar y limpiar los datos
para garantizar su calidad y fiabilidad. Por tanto podemos decir que las
características principales del Big Data pueden agruparse dentro de la
regla de las 4 ``V''.

\hypertarget{aplicaciuxf3n-de-las-tecnologuxedas-al-emprendedor}{%
\subsection{Aplicación de las tecnologías al
emprendedor}\label{aplicaciuxf3n-de-las-tecnologuxedas-al-emprendedor}}

La aplicación de smart data coo tecnología junto con IA, Bid Data y el
reto de las tecnologías a gran escala implica utilizar técnicas y
herramientas avanzadas de análisis de datos para obtener información
relevante y tomar decisiones inteligentes en el contexto de un negocio o
proyecto emprendedor. Antes de empezar, es fundamental tener claridad
sobre los objetivos que se desealograr con el análisis de datos. ¿Se
busca identificar oportunidades de mercado? ¿Optimizar estrategias de
marketing? ¿Mejorar la eficiencia operativa? Establecer objetivos claros
ayudará a enfocar los esfuerzos durante los comienzos de nuevos
proyectos. Una vez se han establecido cuales son los objetivos a lograr
con los medios e información que tenemos se debe tener claro cuales son
las fuentes de datos que se van a usar, es decir, las que son más
relevantes para el negocio. Ya seleccionadas las fuentes hay que
asegurarse que los datos sean claros, precisos y estén actualizados.
Como hemos visto en los anteriores apartados, al tratarse de grandes
cantidades de información para las nuevas tecnologías y sobre todo, las
que son de libre uso como las redes sociales pueden tener datos falsos,
debiendo tener cuidado de no medir con esta información. Estos datos
deben ser recopilados, clasificados y almacenados con herramientas de
gestión de bases de datos o plataformas de almacenamiento en la nube
para facilitar este proceso debiendo asegurarse del cumplimiento de las
diferentes normativas de GDPR, pero a su vez los datos recopilados
pueden contener errores, valores atípicos o información irrelevante. Es
necesario realizar una limpieza y preprocesamiento de los datos para
eliminar inconsistencias y asegurar su calidad. Esto implica tareas como
eliminar registros duplicados, corregir errores y completar datos
faltantes. Es aquí cuando entra en juego la aplicación de técnicas de
análisis como el machine learning o la minería de datos y análisis
estadístico para extraer información valiosa de los datos. Estas
técnicas permitirán identificar patrones, tendencias, correlaciones y
otros conocimientos relevantes para el negocio. Junto con el análisis se
debe hacer una interpretación de los datos mediante gráficos y tablas.
Una vez obtenidos los resultados del análisis, es importante
interpretarlos en el contexto de cada negocio, evaluando los hallazgos
obtenidos y utilizándolos como base para tomar decisiones informadas. La
aplicación de Smart Data es un proceso continuo. A medida que se obtenga
más datos y se realicen nuevos análisis, es importante mejorar los
enfoques, aprendiendo de los resultados obtenidos y ajustando las
estrategias para optimizar la toma de decisiones.

\hypertarget{beneficios-y-limitaciones-del-anuxe1lisis-de-datos-en-el-sector-turuxedstico}{%
\subsection{Beneficios y limitaciones del análisis de datos en el sector
turístico}\label{beneficios-y-limitaciones-del-anuxe1lisis-de-datos-en-el-sector-turuxedstico}}

Como podemos ver, estas herramientas han supuesto un gran avance en el
sector turístico en general, y su aplicación permite potenciar a las
empresas en su crecimiento y optimización, pero al igual que todo, estas
herramientas y métodos tienen su parte positiva y sus beneficios y sus
limitaciones. Por un lado, destacamos como beneficios la gran ayuda que
son a la hora de la toma de decisiones informadas. El análisis de datos
permite a las empresas turísticas tomar decisiones basadas en evidencia
y datos reales. Puede proporcionar información valiosa sobre patrones de
reserva, preferencias de los clientes, tendencias del mercado y
comportamiento del consumidor, lo que permite a las empresas adaptar sus
estrategias y crear campañas con base de conocimiento sólida. La
personalización de servicios es posible gracias a estos métodos. El
análisis de datos permite a las empresas turísticas comprender mejor a
sus clientes y ofrecer servicios más personalizados. Al analizar datos
demográficos, preferencias y comportamiento de reserva, las empresas
pueden adaptar sus ofertas, promociones y recomendaciones para
satisfacer las necesidades individuales de los clientes y mejorar su
experiencia de viaje. También son de gran ayuda en cuanto a la
optimización de precios y ofertas se refiere, permitiendo a las empresas
turísticas ajustar los precios y las ofertas de manera más precisa. Al
analizar datos de mercado, demanda, competencia y otros factores
relevantes, las empresas pueden establecer precios competitivos y
estrategias de precios dinámicos que maximicen los ingresos y la
ocupación de los alojamientos. Por último, debemos destacar la mejora de
la gestión operativa, ayudando a mejorar la eficiencia y la gestión
operativa. Al analizar datos sobre la ocupación de los alojamientos, la
disponibilidad de recursos, la gestión de inventario y otros factores
operativos, las empresas pueden optimizar la asignación de recursos,
reducir costos y mejorar la satisfacción del cliente. También aparecen
limitaciones e inconvenientes que pueden causar dificultades a las
empresas. Por un lado, la calidad de los datos debe ser buena, y en el
sector turístico no siempre ocurre, debido a que pueden estar dispersos
en diferentes plataformas, pueden ser datos falsos (como las opiniones
en Google o Tripadvisor que no tienen control) o pueden contener
errores, lo que dificulta el uso efectivo de los mismos en análisis. Un
aspecto muy importante que se debe tener en cuenta es la privacidad y la
protección de datos de la información de cada cliente. El sector
turístico maneja datos personales sensibles de los clientes, como
información de identificación, preferencias de viaje y detalles de pago.
Es fundamental garantizar la privacidad y la protección de estos datos.
El análisis de datos debe cumplir con las regulaciones de privacidad y
seguridad de datos, como el Reglamento General de Protección de Datos
(GDPR) en la Unión Europea, lo que puede imponer restricciones al acceso
y uso de los datos, pero también debe tener en cuenta las leyes de cada
país, ya que son diferentes. Se requiere además una adecuada
interpretación y contextualización de los resultados. Los datos por sí
solos no proporcionan respuestas definitivas, sino que requieren una
comprensión profunda del dominio turístico y la capacidad de analizar
los resultados en el contexto adecuado. La interpretación errónea de los
datos puede llevar a decisiones equivocadas o conclusiones incorrectas.
No debemos olvidarnos de que el sector turístico es altamente dinámico y
está sujeto a cambios rápidos en las preferencias de los clientes, las
tendencias de viaje y las condiciones del mercado, por lo que cuando se
aplican estas herramientas de análisis se debe tener una actualización
regular de los modelos y algoritmos utilizados para obtener información
relevante y actualizada.

\hypertarget{reflexiuxf3n-sobre-los-desafuxedos-y-las-oportunidades-futuras-en-este-uxe1mbito}{%
\subsection{Reflexión sobre los desafíos y las oportunidades futuras en
este
ámbito}\label{reflexiuxf3n-sobre-los-desafuxedos-y-las-oportunidades-futuras-en-este-uxe1mbito}}

Algo que ha quedado claro es que el turismo es una actividad
interdependiente de otras industrias por lo que, al pausarse esta,
sectores como la movilidad, la logística, la salud, la agricultura, la
automoción o la alimentación, entre otros, se vieron también afectados.
Viendo esto, y si pensamos, por ejemplo, en el caso de España, el PIB
cerró el año 2022 con 159.490 millones de euros, cifra que es un 1,4\%
superior a la del año 2019, según el informe trimestral de Perspectivas
turísticas de la Alianza para la Excelencia Turística (Exceltur). El
sector crece cada vez más y aporta más valor, pero las fuentes de datos
se vuelven más amplias y complicadas de manjar en términos de análisis.
De ahí, que numerosos gobiernos, entre los que destacamos el caso de
España, se han propuesto crear espacios ``inteligentes'' en los cuales
se aporte seguridad a la hora de las empresas compartan sus datos con
otras del sector o en general, por lo que promueven el intercambio y
combinación de datos. La generación de estos espacios de compartición y
explotación de datos supondrá grandes ventajas para el sector, ya que se
facilitará la creación de ofertas, productos y servicios más
personalizados que proporcionen una experiencia mejorada y adaptada a
las necesidades de los clientes, mejorando así la capacidad de atraer
turistas. En términos generales generales y como desafíos podemos
destacar la mejora del tamaño y la calidad de los datos, siendo este uno
de los mayores desafíos ya que la gestión de grandes volúmenes de datos
sigue siendo una complicación y por tanto garantizar su calidad puede
complicarse. La industria turística genera una gran cantidad de
información, desde reservas y transacciones hasta interacciones en redes
sociales y opiniones de los clientes. La recopilación, integración y
limpieza de estos datos de manera eficiente y precisa puede ser un
desafío técnico. En cuanto al cumplimiento del GDPR pueden surgir
complicaciones debido a las diferentes normativas que existen en
diferentes países y marcos legislativos, por lo que deben garantizar que
los datos de los clientes estén protegidos contra accesos no autorizados
o brechas de seguridad. La interpretación correcta de los resultados y
la capacidad de traducirlos en acciones concretas pueden ser
desafiantes. Se requiere una combinación de experiencia en el sector
turístico y habilidades analíticas para aprovechar al máximo el análisis
de datos. A pesar de estos desafíos, también existen oportunidades que
pueden conseguir en el futuro gracias a los desarrollos tecnológicos,
como pueden ser la personalización y experiencia del cliente al
comprender las preferencias individuales de los clientes, las empresas
pueden ofrecer recomendaciones y ofertas personalizadas, lo que mejora
la satisfacción del cliente y fomenta la fidelidad. En cuanto a la
predicción y anticipación de la demanda se verán beneficiados utilizando
técnicas avanzadas de análisis y modelos predictivos, las empresas
turísticas pueden anticipar la demanda futura y ajustar su oferta en
consecuencia. Por otro lado, el uso más reciente pero con mayor auge
será el de las redes sociales, lo que permitirá una lectura correcta de
los comentarios y opiniones además del análisis de las ubicaciones de
los extranjeros y nacionales en un destino puntual para ver los
comportamientos.

\hypertarget{referencias-1}{%
\subsection{Referencias}\label{referencias-1}}

Las fuentes estadísticas y su rentabilidad en el sector turístico.
(s/f). Cehat.com. Recuperado el 8 de junio de 2023, de
https://cehat.com/las-fuentes-estadisticas-y-su-rentabilidad-en-el-sector-turistico/
EP. (2023, enero 17). El PIB turístico cerró 2022 recuperando el nivel
prepandemia. Ediciones EL PAÍS S.L.
https://cincodias.elpais.com/cincodias/2023/01/17/companias/1673960751\_828579.html
Es, D. G. (2023, febrero 7). Radiografía del dataspace nacional de
Turismo: retos y oportunidades para el sector turístico. datos.gob.es.
https://datos.gob.es/es/blog/radiografia-del-dataspace-nacional-de-turismo-retos-y-oportunidades-para-el-sector-turistico
El Sistema de Inteligencia Turística ayuda a los destinos a conocer las
necesidades de los turistas. (2016, junio 2). SEGITTUR.
https://www.segittur.es/sala-de-prensa/notas-de-prensa/el-sistema-de-inteligencia-turistica-ayuda-a-los-destinos-a-conocer-las-necesidades-de-los-turistas/
Sabadell, B. (s/f). El Blog de Banco Sabadell. El Blog de Banco
Sabadell. Recuperado el 8 de junio de 2023, de
https://blog.bancsabadell.com/2018/04/las-cuatro-v-del-big-data-volumen-variedad-velocidad-veracidad-que-es-el-big-data-para-que-sirve-con-sabadell-campus.html
Hosteltur. (2023, marzo 24). Airbnb supera las cifras prepandemia a lo
grande. Hosteltur.
https://www.hosteltur.com/156656\_airbnb-supera-las-cifras-prepandemia-a-lo-grande.html
Hosteltur. (s/f). Las 6 claves del uso del Big Data en el turismo.
Hosteltur: Toda la información de turismo. Recuperado el 8 de junio de
2023, de
https://www.hosteltur.com/comunidad/005340\_las-6-claves-del-uso-del-big-data-en-el-turismo.html
Juan, C. (2023). Las oportunidades de negocio que ofrece el Big Data en
el sector turístico. Thinking for Innovation.
https://www.iebschool.com/blog/big-data-en-el-sector-turistico-big-data/
Braintrust-Ekm, C. (2019, septiembre 19). Why should the tourism company
know its customers? BrainTrust CS.
https://www.braintrust-cs.com/en/empresa-turistica-conocer-clientes/
López, R. G. (2022, junio 30). Tecnologías que transforman el turismo.
Marketing Turístico Digital - Las nuevas tendencias del Marketing
Digital incorporadas al sector turístico.
https://marketingturisticodigital.com/2022/06/30/tecnologia-y-turismo/

\bookmarksetup{startatroot}

\hypertarget{la-gobernanza-de-datos-en-pymes-emprendedoras}{%
\chapter{La Gobernanza de Datos en PYMEs
Emprendedoras}\label{la-gobernanza-de-datos-en-pymes-emprendedoras}}

Autora: Bielka Yuvnny Ulloa Reynoso

Lorem ipsum dolor sit amet, consectetur adipiscing elit. Praesent
dapibus ut libero nec semper. In quam lorem, rutrum in nulla quis,
elementum volutpat odio. Phasellus felis nunc, semper eu sodales eu,
laoreet nec arcu. Sed ac magna quis sapien accumsan gravida. Etiam
tristique dui id elit egestas condimentum. Integer gravida fermentum
placerat. Ut hendrerit viverra ipsum, id vehicula magna tincidunt sed.
Nunc fermentum diam purus, non dignissim purus tincidunt vitae. Sed
tincidunt tortor vitae malesuada molestie. Sed aliquet, est a vulputate
aliquet, massa erat hendrerit arcu, in ultrices nulla nisl vel tortor.
Nam velit est, venenatis et tortor ac, rhoncus feugiat est. Nunc non
neque erat. Etiam eget ipsum fermentum risus lobortis consequat sit amet
sit amet dui. Maecenas auctor vehicula volutpat.

\hypertarget{titular-12}{%
\section{Titular}\label{titular-12}}

Etiam ultricies magna imperdiet nunc malesuada, ac lobortis sapien
rhoncus. Aenean eros lectus, accumsan vitae faucibus a, aliquam ac diam.
Maecenas finibus justo non nibh pharetra aliquam. Maecenas egestas, ex
vitae blandit convallis, leo mauris ornare nunc, porta fermentum tellus
est et urna. Aenean eu est et sapien laoreet placerat. Nunc turpis
ipsum, dapibus a ipsum at, tempus pretium lacus. In libero turpis,
tristique id orci a, auctor rhoncus risus. Praesent lacus nunc,
sollicitudin quis ipsum id, pulvinar venenatis mi.

\hypertarget{titular-13}{%
\subsection{Titular}\label{titular-13}}

Vivamus libero leo, accumsan non turpis vel, eleifend sodales mauris.
Donec pellentesque, tellus a lacinia vestibulum, nisl lacus dapibus
nibh, id tincidunt turpis erat non nisl. Donec ornare imperdiet metus,
eu sollicitudin risus elementum non. Aliquam metus eros, maximus
dignissim pellentesque nec, venenatis vitae purus. Vivamus laoreet mi
magna, ac malesuada tellus condimentum in. Integer nulla lectus, finibus
sit amet ex nec, finibus molestie diam. Pellentesque gravida ac erat sit
amet tempus.

\bookmarksetup{startatroot}

\hypertarget{references}{%
\chapter*{References}\label{references}}
\addcontentsline{toc}{chapter}{References}

\markboth{References}{References}

\hypertarget{refs}{}
\begin{CSLReferences}{0}{0}
\end{CSLReferences}



\end{document}
